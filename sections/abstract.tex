\documentclass[../main.tex]{subfiles} 
\begin{document}   

Web Servers are now often deployed in the cloud environment. To secure
communications in the middle of the network between clients and web
servers, the Internet community has come to rely on the SSL/TLS
protocol which requires a strong secret (private key) to be stored
with the server. However, the server administrator in this model is
forced to trust the cloud provider to not disclose their private key.
We propose a modification to legacy web servers and cryptographic
libraries which secures the sensitive key material in the face of a
malicious provider, a compromised operating system, or a compromised
hypervisor, without the need to trust the cloud provider, operating
system, and/or hypervisor. 

To this end, we designed a system that adopts the principle-of-least
privelege, the idea that each component of a system should have access
to the minimum resources required for its functioning, and a
technology from \Intel~called Software Guard Extensions (SGX) to
isolate the private key. SGX is an extension to \Intel's ISA which
enables a developer to create protected regions of memory, called an
enclave, that may only be accessed by the program that initialised it,
the \textit{enclave program}. We also developed a second design
that secures the symmetric session key

We split a legacy web server into two components a, low privelege,
untrusted component that cannot access the private key directly, and a
high privelege componenet that has ownership of the private key. The
high privelege component is an \textit{enclave program}, and we use
SGX to create an enclave where we place the private key via a process
called remote-attestation, which permits the developer to place
secrets in an enclave, present on a remote machine, in a secure
fashion. We carefully define an interface between the untrusted
componenet and the \textit{enclave program}, to permit the
establishment of SSL/TLS sessions while maintaining the integrity of
the private key. The private key is only used to establish a session
in SSL/TLS. Communications between a client and the server are secured
by a symmetric session key that is negotiated in the session
establishment step. Consequently, the interface we define is only
invoked when establishing a session.

Note, however, the aforementioned design only isolates the private
key. In fact, the symmetric session keys are left to the untrusted
component in that design. An attacker, by exploiting the untrusted
component, can obtain the session keys for all on-going sessions and
store them for later use. Additionally, by eavesdropping on the
communication between the client and the server, he can capture all
the encrypted payloads exchanged. The assailant can then decrypt, on
another machine, all of these stored payloads, using the session keys
acquired, and access sensitive user information. We came up with a
second design that makes this attack more difficult.

The second design places the session keys inside the enclave, and
allows the untrusted component to access them through an
encrypt/decrypt interface. While this is an \textit{oracle} for the
session keys, it is no longer possible for an attacker to acquire the
session keys and store them to decrypt prior, stored communication.
The attacker must now actively exploit the server to decrypt payloads,
and may only decrypt payloads for currently ongoing sessions, a more
challenging task when compared to the attack described above.

In creating these designs, we aimed to minimise the performance
penalty incurred due to the execution of additional SGX instructions
and the partitioning of the web server application. We conducted an
evaluation of a prototype that we implemented using NGINX, a modern
web server, LibreSSL, an SSL/TLS library, and OpenSGX, an emulator for
the SGX instruction set (we used an emulator because we could not
obtain SGX hardware). Our end-to-end performance evaluation
demonstrates that the performance penalty exhibited by our prototype
is 18\% for SSL/TLS cipher suites that do not offer forward secrecy,
and 30\% for cipher suites that provide forward secrecy for the design
where session keys are in the untrusted component. When session keys
are held by the \textit{enclave program}, the performance pelt is 29\%
for cipher suites that do not offer forward secrecy, and 37\% for
cipher suites that provide forward secrecy. While the performance
overhead due to the added security, according to our evaluation, is
significant, it may be a lucrative trade-off for web server
administrators who wish to secure their long-term private key against
the threats mentioned above.


\end{document}
