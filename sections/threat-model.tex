\documentclass[../main.tex]{subfiles}

\begin{document}
Before presenting our design, we introduce the threat model we
utilised to scope the discussion. We consider a threat model similar
to the one presented in Haven~\cite{Baumann14}. Our TCB includes a
correctly implemented and SGX enabled CPU and all instructions
executing and data resident within an enclave. Therefore, we assume an
attacker cannot access the SGX processor key provisioned by \Intel
itself, which is used to generate subsequent cryptographic keys that
preserve the confidentiality, integrity and authenticity of an
enclave.

An adversary can take full control of everything beyond the TCB. That
is, we assume all software executing on the platform (outside of the
enclave), the underlying operating system, the hypervisor, all
firmware and the BIOS are potentially compromised. Side-channel
attacks that originate from other sources such as CPU cache timing
information (L1, L2, L3), power consumption or other entropy source
are considered as out of scope of this work. We also assume an
attacker may act as man-in-the-middle to eavesdrop active sessions
and as further analysed in Section \ref{sec:limitations}, launch
a denial-of-service attack, though without compromising any secret
isolation guarantees of our design.

As further described in Section \ref{sec:implementation}, our design
uses a single enclave where state for all active SSL connections is
stored in memory. If an adversary compromises the code running within
the trusted component, security parameters such as session keys could
be owned as well. Note that, this would not be of great interest as
in this case the private key itself is exposed as well. Thus, it
would be trivial for the attacker to generate all past and current
connections' session keys for ciphers with no forward secrecy support
and all current connection's session keys for ciphers with forward
secrecy support. Nevertheless, we consider adapting any software
fault isolation scheme within the trusted component as out of scope
of this work and we assume all code running within the enclave is
trusted.
\end{document}

%%% Local Variables:
%%% mode: latex
%%% TeX-master: "../main"
%%% End:
