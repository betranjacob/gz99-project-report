\documentclass[../main.tex]{subfiles}
 
\begin{document}
Securing the private-key against key against an adversary capable of
exploiting the operating system is achievable through use
of specialised hardware called a trusted platform module (TPM), a
device that can secure the private key through encryption via a
Storage Root Key (SRK)~\cite{tpm10}. The SRK's integrity is maintained
by ensuring that its private component may never leave the TPM. As a
result, the long term private key itself can never be decrypted
outside the TPM. The TPM is also capable of executing cryptographic
operations, including those that make use of the long term private key
in SSL/TLS. Consequentially, by delegating all private key operations
to the TPM, one can rest assured that their private key cannot be
compromised without compromising the TPM itself.

The security benefits of a TPM, however, were outweighed by the cost
of purchasing the additional piece of hardware, and TPMs did not gain
any traction with cloud providers. In this project we utilized a new
technology from \Intel called Software Guard Extensions (SGX). SGX is
an augmentation to \Intel's ISA which offer the ability to launch
encrypted regions of memory, called enclaves, where only trusted
regions of code can read/write. This allows for TPM like
functionality, but SGX has the advantage of being deployed as part of
new CPUs released by \Intel and as a result, when cloud providers
upgrade their machines they will possess the ability to support SGX
programs without purchasing additional hardware.

% This might not have to be placed here, but I guess its doing no harm
% being here
SGX has gained momentum as a research platform for security related
work such as Haven~\cite{Baumann14} which secures a legacy application
from an non-trusted OS and cloud-provider \textit{without} modifying
the application's source code.  Yet, there is no work, to our
knowledge, that attempts to secure only the private key material
through use of SGX. Narrowing the trusted region to contain only the
component that handles the private key allows us to define a much
smaller trusted computing base that only contains the CPU and the code
that handles the long term private key. 

\end{document}