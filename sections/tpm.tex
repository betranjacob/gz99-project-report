\documentclass[../main.tex]{subfiles}
 
\begin{document}
Securing the private key against an adversary capable of exploiting
the operating system is achievable through use of specialised hardware
called a trusted platform module (TPM). TPMs can secure the private
key through encryption via a Storage Root Key (SRK)~\cite{tpm10}. The
SRK's integrity is maintained by ensuring that its private component
may never leave the TPM. As a result, the long term private key itself
can never be decrypted outside the TPM. The TPM is also capable of
executing cryptographic operations, including those that make use of
the long term private key in SSL/TLS. Consequentially, by delegating
all private key operations to the TPM, one can rest assured that their
private key cannot be compromised without compromising the TPM itself.

However, cloud-providers today do not make use of TPMs for SSL/TLS. In
this project we used a new technology from \Intel~called Software
Guard Extensions (SGX). SGX is an augmentation to \Intel's ISA which
offers the ability to launch encrypted regions of memory, called
enclaves, where only trusted regions of code can read/write. This
allows for TPM-like functionality, but SGX has the advantages of:
\begin{enumerate}
  \item Permitting the execution of arbitrary code within an enclave whereas a
    TPM only supports cryptographic operation
  \item being deployed as part of new commodity CPUs released by \Intel.
    As a result, cloud-providers will be able to support SGX programs
    as they upgrade their machines.
\end{enumerate}

% This might not have to be placed here, but I guess its doing no harm
% being here

\end{document}