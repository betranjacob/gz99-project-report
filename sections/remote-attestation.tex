\documentclass[../main.tex]{subfiles}

\begin{document}
\subsection{Provisioning the enclave with the long-term private key}
Provisioning a web server with a long-term private key for the purposes of SSL/TLS is currently done storing the private-key along with the executable on 
the remote machine. This scheme, however, assumes a trusted cloud-provider/OS. However, under our threat model this scheme is not viable. Alternatively,
we require a method by which we can verify the identity and integrity of the server application and then, upon successful verification, we can 
send the long term private key to the server in a secure fashion. Loosely, the requirements are as follows:
\begin{itemize}
	\item The mechanism allows the verification of the identity and integrity of the server application and the underlying TCB.
	      This is so that we can be sure the private-key is being sent to the same server we placed on the remote machine, and the software is being executed by
	      trustworthy hardware.
	\item The mechanism allows us to setup a secure channel, ensuring that the only entities privy to the private key 
	      are the server application and the server administrator.
\end{itemize} 
The first and second requirement are met by a process called inter-platform attestation.
\subsection{Inter-platform attestation} %reference(https://software.intel.com/en-us/articles/innovative-technology-for-cpu-based-attestation-and-sealing)
Inter-platform attestation is a mechanism that can be invoked by an entity, referred to as the challenger, running on one platform
to verify an enclave running on another, remote, platform. This process is enables the challenger to verify the following about the remote enclave:
\begin{enumerate}
	\item The contents of the enclave's pages (code, data, stack and heap) upon creation (after the ECREATE instruction completes)
	\item The identity of the entity that signed the enclave 
	\item The trustworthiness of the underlying hardware
	\item Authenticity and integrity of any data generated by the enclave and sent as part of the attestation process. This allows us to satisfy the second requirement by generating an ephemeral key
	      pair and binding it to the remote attestation process. This, therefore, allows the challenger to verify the integrity of the ephemeral public key and verify that it was generated by the server application.
\end{enumerate}
The steps involved in the attestation process are as follows (illustrated in Figure~\ref{fig:attest}):
\begin{enumerate}
	\item The challenger invokes the remote attestation mechanism to verify the identity and integrity of the remote enclave
	\item The non-trusted part of the web server receives the challenge, passes it along to the trusted portion of the web server
	      along with the identity of the quoting enclave. The quoting enclave is a special enclave provided by Intel as part of the SGX
	      platform to enable remote attestation by verifying the integrity of the underlying hardware. %refer readers to Intel white paper  
	\item The enclave invokes EREPORT which is an SGX instruction that generates a REPORT structure to be provided to a \textit{local} enclave, the quoting enclave in this case.
	      This structure contains a hash of the contents of the enclave's pages upon ECREATE's termination, a hash of the identity of the enclave's signer, a hash of any user-data, the ephemeral key in our case,
	      generated by the enclave. The REPORT is signed by a MAC-key that can only be accessed by the CPU and the quoting enclave. The REPORT along with the ephemeral key is then sent to the non-trusted part of the application.
	\item The REPORT is sent to the quoting enclave where its integrity is verified by calculating the MAC across its contents. 
	\item Assuming the REPORT is verified successfully, the quoting enclave generates a QUOTE structure that includes the REPORT structure and a signature across
	      the quote generated using a key known as the EPID key. %add expansion of EPID
	      The EPID key is a private key unique to the CPU that is part of the platform and verifies the firmware of the processor and its SGX capabilities. 
	\item The QUOTE is sent along with the ephemeral key to the challenger
	\item The challenger verifies the QUOTE structure by using an EPID public certificate. If this is successful then the challenger is sure that this QUOTE came from a valid SGX CPU and can trust
	      its authenticity. The challenger can then check the contents of the REPORT contained within the QUOTE to verify the identity of the remote enclave, and the integrity of the ephemeral key 
	      received along with the QUOTE. The ephemeral key, if proven to be valid, can now be used to communicate with the remote enclave in a secure manner.  
\end{enumerate}
		
\begin{figure}[H]
	\centering
	\includegraphics[scale=0.7]{attestation-sealing.jpg}
	\caption{Remote Attestation and Secret Provisioning}
	\label{fig:attest}
\end{figure}


\end{document}