\documentclass[../../../main.tex]{subfiles}

\begin{document}
\label{sec:mem-analysis}
SGX incorporates a hardware memory encryption engine (MEE) that
encrypts any data that resides in an enclave's reserved area in RAM.
The MEE also decrypts data that is required by the CPU during the
course of executing an \textit{enclave program}. Nevertheless, the MEE
is not used on data in the CPU's cache. As a result, an
\textit{enclave program} incurs additional overhead whenever the CPU
cannot find the data required for an instruction in the cache. This is
because a cache miss results in two things:
\begin{enumerate}
  \item Allocate space in the CPU cache for the needed data. This may entail
    the removal of data from the CPU cache. Data removed, if it belongs
    to an enclave and has been modified, may require encryption by the MEE.
  \item Decryption of the requested data, incoming from main memory, by the MEE.
\end{enumerate}

\subsubsection*{Test Setup}
To estimate the impact of the MEE, we implemented a second version of
our \textit{enclave program} (enclave model) that has no dependencies
on OpenSGX (this modelling approach was discussed in
Section~\ref{sec:perfmodel}). We utilised an application from
\Intel~called \VTune~Amplifier XE 2016. \VTune amplifier captures the
number of CPU cycles required to execute a program and the number of
CPU cycles wasted waiting on memory due to last level cache (LLC)
misses \footnote{We are only concerned with LLC misses because only
  those misses lead to accessing main memory.}. Given those two
numbers, we can calculate the percentage of CPU cycles wasted in
accessing main memory. This result can then be used as a proxy for
estimating the overhead of the MEE's execution. However, we need to
saturate the web server (NGINX), and by extension the enclave movel,
with requests to produce a reliable estimate. To this end, we utilised
\Apache~Benchmark a utility program that emulates web server workloads
by specifying a number of clients that constantly generate HTTP(S)
requests, to flood NGINX with requests.

We ran the enclave model, within \VTune~Amplifier, and NGINX on a
personal laptop. The latter comprised of an \Intel~Core i5 2520M @
2.50GHz with 3MB of L3 cache and 16GB of RAM @ 1600MHz. We used
\Apache~Benchmark with ten concurrent connections, producing a total
of 1000 HTTPS requests/s, to NGINX. We ran the test four times:
\begin{enumerate}
  \item Using a cipher lacking support for forward secrecy,
    RSA-AES256-SHA, with the design where session keys reside in the
    untrusted component
  \item Using a cipher lacking support for forward secrecy,
    RSA-AES256-GCM-SHA384, with the design where session keys reside in the
    enclave
  \item Using a cipher offering support for forward secrecy,
    ECDHE-AES256-GCM-SHA384, with the design where session keys reside in
    the untrusted component
  \item Using a cipher offering support for forward secrecy,
    ECDHE-AES256-GCM-SHA384, with the design where session keys reside in
    the enclave
\end{enumerate}

\subsubsection*{Results}
The values reported from \VTune~ Amplifier for the above-mentioned
tests are in Table~\ref{tab:mem}. We used the equation
below~\cite{intel-eqn} to calculate the percentage of cycles the
enclave program wasted on waiting for cache misses\footnote{According
  to \Intel, each memory load from DRAM, on an i5-2520M processor,
  consumes approximately 180 cycles}.

\begin{align*}
 \text{\% of cycles spent on memory access} = 
    \frac{\text{CPU Cycles Wasted On LLC Misses} * 180}{\text{Total CPU
    Cycles}}
\end{align*}

\begin{table}[H]
  \resizebox{\linewidth}{!}{ \pgfplotstabletypeset[assign column
    name/.style={/pgfplots/table/column name={\textbf{#1}}},col
    sep=comma, header=has colnames, columns/{CPU Cycles Wasted}/.style
    = {column name = {CPU Cycles Wasted(\%)}}]{results/cache_analysis.csv}}
  \caption{Memory Analysis}
  \label{tab:mem}
\end{table}

\subsubsection*{Analysis of Results}
Before running this microbenchmark, we expected to see little memory
 accesses on key-outside implementation because it only process ssl
 handshake. All data should be able to fit in the cache and not require
 memory access. From the test result, we found that memory access rate can 
 range from 0\% to 0.11\% of total cycles spent.  In the best case, we may 
 have no overhead from memory encryption because all data fit in cache. 
 Testing from Intel[REFERENCE] indicate that application can slowdown from 
 5\% to 14\%. Intel tuning guide suggest that typical programs have cache 
 misses at 0.2\%. Therefore, we estimate that our implementation could slow 
 down for up to 9.5\% based on the worst case that our implementation has 
 cache miss rate 55\% of typical application .
\end{document}

%%% Local Variables:
%%% mode: latex
%%% TeX-master: "../../../main"
%%% End:
