\documentclass[../../../main.tex]{subfiles}

\newcommand\numberthis{\addtocounter{equation}{1}\tag{\theequation}}

\begin{document}
\label{sec:mem-analysis}
SGX incorporates a hardware memory encryption engine (MEE) that
encrypts any data that resides in an enclave's reserved area in DRAM.
The MEE also decrypts data that is required by the CPU during the
course of executing an \enclaveprogram. Nevertheless, the MEE is not
used on data in the CPU's cache. As a result, an
\enclaveprogram~incurs additional overhead whenever the CPU cannot
find an instruction's operand in the cache. The overhead of the MEE is
due to the following:
\begin{enumerate}
  \item Allocating space in the CPU cache for the needed data. This
    may entail evicting data from the CPU cache. Data removed, if it
    belonged to an enclave and has been modified, may require encryption
    by the MEE.
  \item Decryption of the requested data, incoming from main memory
    (DRAM), by the MEE.
\end{enumerate}

To estimate the impact of the MEE we measured the number of last level
cache (LLC) misses as only those operations lead to accessing the main
memory (DRAM). We utilised an application from \Intel~called
\VTune~Amplifier XE 2016. It captures the number of CPU cycles
required to execute a program and the number of LLC misses when
executing a \texttt{load} instruction. Given those two numbers, the
fraction of the \enclaveprogram~execution (in CPU cycles) wasted on
accessing the main memory can be calculated using the equation
below~\cite{intel-eqn} (according to \Intel, each memory load from DRAM
on an i5-2520M processor consumes approximately 180 CPU cycles).

\begin{align*}
  \text{Fraction of CPU Cycles Wasted} = \numberthis \label{eqn1}
  \frac{\text{Number of LLC Misses} * 180} 
  {\text{Total CPU Cycles}}
\end{align*}


\subsubsection*{Test Setup}
We ran the \enclavemodel~within \VTune~Amplifier on a personal laptop
that comprised of an \Intel~Core i5 2520M @ 2.50GHz with 3MB of L3
cache (LLC) and 16GB of RAM @ 1600MHz. Note that this set-up is not
the same as the one used to evaluate the end-to-end performance
described in Section~\ref{sec:endtoend} since \VTune~Amplifier needs
access to CPU registers not available in the virtual machine
environment. The main difference is the amount of available LLC which
is 10x larger (30 MB) on the machine hosted in the AWS cloud.

We saturate the web server (NGINX), and by extension the \enclavemodel, with
requests to produce a reliable estimate. For this task we used
\Apache~Benchmark, a utility program that simulates web server workloads. We
set up ten concurrent connections, resulting in a total of 1000 HTTPS
requests/s to NGINX.

We ran the test eight times, requesting 1 KB and 32 MB objects for each of the
cases below:
\begin{enumerate}
  \item Using a cipher lacking support for forward secrecy,
    AES256-GCM-SHA384, with the design where session keys reside
    \textbf{outside} the trusted component
  \item Using a cipher lacking support for forward secrecy,
    AES256-GCM-SHA384, with the design where session keys reside
    \textbf{inside} the trusted component
  \item Using a cipher offering support for forward secrecy,
    ECDHE-RSA-AES256-GCM-SHA384, with the design where session keys reside
    \textbf{outside} the trusted component
  \item Using a cipher offering support for forward secrecy,
    ECDHE-RSA-AES256-GCM-SHA384, with the design where session keys reside
    \textbf{inside} the trusted component
\end{enumerate}

We test the request of the (quite unreasonable) 32 MB object to force
our \enclavemodel~to access the DRAM and place an upper bound on the
performance hit due to the MEE.

\subsubsection*{Results}
The values obtained from \VTune~Amplifier for the above-mentioned
tests are reported in Table~\ref{tab:mem}. Note that these concern
only the \enclavemodel~execution and do not include the execution of
untrusted component, because only the \enclaveprogram~suffers from the
MEE overhead. The CPU Cycles Wasted column is calculated according to
the equation~\eqref{eqn1} above.

\begin{table}[H]
  \resizebox{\linewidth}{!}{
    \pgfplotstabletypeset[
     assign column name/.style={/pgfplots/table/column name={\textbf{#1}}},
     col sep=comma,
     header=has colnames,
     columns/{CPU Cycles Wasted}/.style = {
       numeric type,
       precision=3,
       column name = {CPU Cycles Wasted(\%)}
     },
     columns/{Number of LLC misses}/.style = {
       numeric type,
       precision = 0,
       preproc/expr={##1/180},fixed, fixed zerofill}
    ]{results/cache_analysis.csv}
  }
  \caption{Proportion of CPU cycles wasted on \texttt{load} instructions}
  \label{tab:mem}
\end{table}

\subsubsection*{Analysis of Results}
Given the results in Table~\ref{tab:mem} we can extrapolate the MEE's
overhead by making the following assumptions:
\begin{itemize}
  \item Each \texttt{load} instruction causes the MEE to decrypt data.
  \item We assume that the MEE requires 0.77 cycles to decrypt a byte of data.
    This assumption comes from \cite{cyclesaes} where it was found that
    AES in GCM mode consumes 0.77 cycles per byte of data, and, according to
    ~\Intel, the MEE uses a ``tweaked'' version of AES in counter mode.
  \item We assume that each \texttt{load} instruction fetches a page
    of DRAM (4K bytes). This assumption may be too pessimistic, but CPUs
    tend to pre-fetch additional data to ameliorate the cost of accessing
    DRAM.
\end{itemize}
The above assumptions result in a simple equation to calculate the number
of cycles consumed by the MEE:
\begin{equation*}
  \text{MEE Overhead} = \frac{\text{Number of LLC Misses
    * (180 + 4096 * 0.77)}}{\text{Total Number of CPU Cycles}}
\end{equation*}
Table~\ref{tab:mem2} shows the estimate of the MEE overhead for each
of our test scenarios. As can be seen from the table, the maximum
overhead is 2.04\% even with the pessimistic assumption about the
number of bytes read per \texttt{load} instruction. This number seems
reasonable because the only memory object that does not fit completely
within the cache is the 32MB payload, and a modern CPU will attempt to
execute another runnable process to avoid stalling on memory
operations. However, the MEE is only one of many contributors to the
overall overhead of SGX.

\begin{table}[H]
  \resizebox{\linewidth}{!}{ \pgfplotstabletypeset[assign column
    name/.style={/pgfplots/table/column name={\textbf{#1}}},col
    sep=comma, header=has colnames, columns/{CPU Cycles Wasted}/.style
    = {numeric type, column name = {MEE Overhead (\%)}, fixed, fixed
      zerofill,precision = 3, preproc/expr={##1*(1+(0.77*4096/180))}},
    columns/{Number of LLC misses}/.style = {numeric type, precision =
      0, preproc/expr={##1/180},fixed, fixed zerofill}
      ]{results/cache_analysis.csv}}
  \caption{Approximated MEE overhead}
  \label{tab:mem2}
\end{table}

\end{document}

%%% Local Variables:
%%% mode: latex
%%% TeX-master: "../../../main"
%%% End:
