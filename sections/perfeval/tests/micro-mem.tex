\documentclass[../../../main.tex]{subfiles}

\begin{document}
\label{sec:mem-analysis}
SGX incorporates a hardware memory encryption engine (MEE) that encrypts any
data that resides in an enclave's reserved area in DRAM. The MEE also decrypts
data that is required by the CPU during the course of executing an
\enclaveprogram. Nevertheless, the MEE is not used on data in the CPU's cache.
As a result, an \enclaveprogram~incurs additional overhead whenever the CPU
cannot find the data required for an instruction in the cache. This is because
a cache miss results in two things:
\begin{enumerate}
  \item Allocating space in the CPU cache for the needed data. This may entail
    removal of data from the CPU cache. Data removed, if it belonged to an
    enclave and has been modified, may require encryption by the MEE.
  \item Decryption of the requested data, incoming from main memory (DRAM), by
    the MEE.
\end{enumerate}

To estimate the impact of the MEE we decided to measure the number of last
level cache (LLC) misses as only those operations lead to accessing the main
memory (DRAM).

We utilised an application from \Intel~called \VTune~Amplifier XE 2016. It
captures the number of CPU cycles required to execute a program and the number
of LLC missed load operations. 

Given those two numbers, the fraction of the \enclaveprogram~execution (in CPU
cycles) wasted on accessing the main memory can be calculated using the
equation below~\cite{intel-eqn}

\begin{align*}
 \text{fraction of CPU Cycles Wasted} = 
    \frac{\text{LLC missed load operations} * 180\footnotemark}
      {\text{Total CPU Cycles}}
\end{align*}

\footnotetext{According to \Intel, each memory load from DRAM, on an i5-2520M
processor, consumes approximately 180 cycles.}

We combine this fraction obtained for our \busywait~with the MEE overhead
results reported by \Intel~\cite{cryptoeprint:2016:204}, i.e.\ 5\% to 14\%
slowdown in DRAM access, to estimate the overhead of the MEE's execution in
our program.

\subsubsection*{Test Setup}
We ran the \enclavemodel~within \VTune~Amplifier on a personal laptop that
comprised of an \Intel~Core i5 2520M @ 2.50GHz with 3MB of L3 cache (LLC) and
16GB of RAM @ 1600MHz. Note that this set-up is not the same as the one used to
evaluate the end-to-end performance described in Section~\ref{sec:endtoend}
since \VTune~Amplifier needs access to CPU registers not available in the
virtual machine environment. The main difference is the amount of available
LLC which is 10x larger (30 MB) on the machine hosted in the AWS cloud.

We saturate the web server (NGINX), and by extension the \enclavemodel, with
requests to produce a reliable estimate. For this task we used
\Apache~Benchmark, a utility program that simulates web server workloads. We
set up ten concurrent connections, resulting in a total of 1000 HTTPS
requests/s to NGINX.

We ran the test eight times, requesting 1 KB and 32 MB objects for each of the
cases below:
\begin{enumerate}
  \item Using AES256-GCM-SHA384 cipher, lacking support for forward secrecy,
    with the design where session keys reside in the untrusted component.
  \item Using AES256-GCM-SHA384 cipher, with the design where session keys
    reside inside the enclave.
  \item Using ECDHE-RSA-AES256-GCM-SHA384, cipher offering forward secrecy, with
    the design where session keys reside in the untrusted component.
  \item Using ECDHE-RSA-AES256-GCM-SHA384, with the design where session keys
    reside inside the enclave.
\end{enumerate}

We test the request of the (quite unreasonable) 32 MB object to force our
\enclavemodel~to access the DRAM and place an upper bound on the performance
hit due to the MEE.

\subsubsection*{Results}
The values obtained from \VTune~Amplifier for the above-mentioned tests are
reported in Table~\ref{tab:mem}. Note that these concern only the
\enclavemodel~execution and do not include the execution of untrusted
component. This is because only the \enclaveprogram~suffers from MEE overhead.
The number of cycles waiting on LLC misses is calculated from
\textit{LOAD\_UOPS\_MISC\_RETIRED.LLC\_MISS} metric as presented in the
nominator of the equation above and the CPU Cycles Wasted column is calculated
according to the full equation.

\begin{table}[H]
  \resizebox{\linewidth}{!}{ \pgfplotstabletypeset[assign column
    name/.style={/pgfplots/table/column name={\textbf{#1}}},col
    sep=comma, header=has colnames, columns/{CPU Cycles Wasted}/.style
    = {column name = {CPU Cycles Wasted (\%)}}]{results/cache_analysis.csv}}
  \caption{Memory Analysis}
  \label{tab:mem}
\end{table}

\subsubsection*{Analysis of Results}
Before running this micro benchmark, we expected to see little memory
accesses for the keys outside design because it only instruments the SSL
handshake, which does not involve large amount of data handling. All data
should, therefore, be able to fit in the cache and not require DRAM access.

The results confirm our expectation for the design, where session keys reside
outside the enclave. They also show that small objects are successfully cached
and no CPU time\footnote{By CPU time we mean the total number of instructions
required to execute our program during a test.} is wasted even for the more
expensive design, where session keys reside within the enclave.

The results also show that when handling large objects inside the enclave up
to 0.2 \% of the \enclavemodel~CPU time is spent waiting on LLC misses.
Incorporating the results from~\cite{cryptoeprint:2016:204}, in the worst case
our \enclaveprogram~would spend between 0.21 \% - 0.23 \% of its CPU time on
fetching data from DRAM. Note, that in the cloud environment this would have
been much smaller due to the much larger cache size available.

In the best case, when all the data fits in cache, we may have negligible or
no overhead from memory encryption/decryption.

We also note that \Intel~tuning guide~\cite{intel-eqn} suggests typical
programs experience up to 0.2 \% of CPU time on LLC misses and the performance
hit need not be investigated unless this is considerably larger.
\end{document}

%%% Local Variables:
%%% mode: latex
%%% TeX-master: "../../../main"
%%% End:
