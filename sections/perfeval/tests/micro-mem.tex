\documentclass[../../../main.tex]{subfiles}

\begin{document}
SGX incorporates a hardware memory encryption engine (MEE) that
encrypts any data that resides in an enclave's reserved area in RAM.
The MEE also decrypts data that is required by the CPU during the
course of executing an \textit{enclave program}. Nevertheless, the MEE
is not used on data in the CPU's cache. As a result, an
\textit{enclave program} incurs additional overhead whenever the CPU
cannot find the data required for an instruction in the cache. This is because
a cache miss results in two things:
\begin{enumerate}
  \item Allocate space in the CPU cache for the needed data. This may entail
    the removal of data from the CPU cache. Data removed, if it belongs
    to an enclave and has been modified, may require encryption by the MEE.
  \item Decryption of the requested data, incoming from main memory, by the MEE.
\end{enumerate}

\subsubsection*{Test Setup}
To estimate the impact of the MEE, we implemented a second version of
our \textit{enclave program} (enclave model) that has no dependencies on OpenSGX (this
modelling approach was discussed in Section~\ref{sec:perfmodel}). We
utilised an application from \Intel~called
\VTune~Amplifier XE 2016. VTune
amplifier captures the number of CPU cycles required to execute a
program and the number of CPU cycles wasted waiting on memory due to
last level cache (LLC) misses \footnote{We are only concerned with LLC misses
  because only those misses lead to accessing main memory.}. Given those
two numbers, we can calculate the percentage of CPU cycles wasted in accessing
main memory. This result can then be used as a proxy for estimating the overhead
of the MEE's execution.

We ran the enclave model within \VTune~Amplifier and used \Apache
Benchmark, a utility program that emulates web server workloads by
generating a configurable number of HTT,

\subsubsection*{Results}

We have used the equation from \Intel~to convert the number of LLC misses to
the percentage of cycles the enclave program spent on waiting for
cache misses. Each memory load from DRAM takes approximately 180
cycles.

\begin{align*}
  &\text{Percentage of cycles spent on memory access} =
  \frac{\text{CPU Cycles Wasted On LLC Misses} * 180}{\text{Total CPU
  Cycles}} \\
  &\text{Percentage of Cycles Wasted - Sessions Keys Outside Enclave} = 0.123\% \\
  &\text{Percentage of Cycles Wasted - Sessions Keys Inside Enclave} = 0.118\%        
\end{align*}

\subsubsection*{Analysis of Results}
\end{document}

%%% Local Variables:
%%% mode: latex
%%% TeX-master: "../../../main"
%%% End:
