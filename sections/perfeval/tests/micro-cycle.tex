\documentclass[../../../main.tex]{subfiles}

\begin{document}
\label{sec:cpu-instr-analysis}
This test is designed to measure the overhead of the enclave implementation
against the stock version. The benchmark consists of two parts.  In the first
part, we count the number of instructions for NGINX with model implementation
to evaluate the cost our design. After the results from model implementation
wereobtained, we run the \textit{enclave program} in OpenSGX to emulate the SGX
instruction set.

\subsubsection*{Test Setup}
In the first test, we instrument NGINX with model enclave implementation using
\textit{Perf}. Perf is a standard linux tool to measure various performance
counters. The requests are generated from \Apache~Benchmark. Due to limitation
of OpenSGX performance, we run only 50 requests for each cipher with 2
concurrent connections.

In the second test, the enclave is run on OpenSGX platform to emulate SGX
instructions. OpenSGX is modified to display application statistics of executed
SGX instructions. We have to ignore the cost of the SGX instructions during
enclave startup, because these instructions are executed only once to prepare
the SGX environment and do not affect the application's response rate in the
long run. Therefore, we have to measure SGX instructions that run on each
request.

\subsubsection*{Results}
Table \ref{tab:cpu-instr-count} shows the number of instructions that were
combined from both parts of this test. The number of executed instructions
in NGINX and in the \textit{enclave program} were obtained from \textit{perf}.
The detailed SGX instruction count in Table \ref{tab:sgx-encls} are executed
only once during startup. Table \ref{tab:sgx-enclu}, shows the SGX instructions
executed in each request. Results for both Tables \ref{tab:sgx-encls} and
\ref{tab:sgx-enclu} were obtained from OpenSGX.

\begin{table}[H]
  \resizebox{\linewidth}{!}{ \pgfplotstabletypeset[assign column
    name/.style={/pgfplots/table/column name={\textbf{#1}}},col
    sep=comma, header=has colnames]{results/instruction_count.csv}}
  \caption{CPU Instruction Count Analysis}
  \label{tab:cpu-instr-count}
\end{table}

\begin{table}[H]
  \center
  \footnotesize
  \pgfplotstabletypeset[assign column
    name/.style={/pgfplots/table/column name={\textbf{#1}}},col
    sep=comma, header=has colnames]{results/sgx_detail_instruction_count_s.csv}
  \caption{SGX Privileged Instructions Count}
  \label{tab:sgx-encls}
\end{table}

\begin{table}[H]
  \resizebox{\linewidth}{!}{ \pgfplotstabletypeset[assign column
    name/.style={/pgfplots/table/column name={\textbf{#1}}},col
    sep=comma, header=has colnames]{results/sgx_detail_instruction_count_u.csv}}
  \caption{SGX Unprivileged User Instructions Count}
  \label{tab:sgx-enclu}
\end{table}

\subsubsection*{Analysis of Results}
As shown in Table \ref{tab:cpu-instr-count}, our implementation added 14.5\% of
executed instructions regarding the design where the session keys reside
outside of the enclave and 16.5\% regarding the design with session keys
reside inside the enclave. While the difference between the number of executed
instructions between the keys-outside and keys-inside are small, the proportion
of instructions executed between the untrusted and trusted component are very
different. Only 4.2\% of instructions were executed in the enclave with the
key-outside implementation but there were 97.7\% of instructions executed in
the enclave for the key-inside implementation. There could be a performance
impact on the key-inside compared to the key-outside implementation, because of
the effect of memory encryption in the enclave, as discussed in Section
\ref{sec:mem-analysis}.
\end{document}
