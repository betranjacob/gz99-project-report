\documentclass[../../../main.tex]{subfiles}

\begin{document}
We conducted this analysis to uncover the impact of our design and SGX
instructions on end-to-end web server performance. To this end, we
utilised the performance model, while varying the number of busywait
cycles, to approximate the overhead of SGX instructions. Additionally,
we evaluated the effect of response size on end-to-end performance by
varying the size of the response to a client's request.

\subsubsection*{Test Setup}
We run our tests on two Amazon EC2 m4.large instances, one for the server and
the other for the client, located on different virtual machines within the same
data center. These machines are comprised of 2 virtual CPUs with 8 GB RAM and
and, according to \textit{iperf} (a network diagnostics tool), the network
bandwidth between the two machines is 500 Mbits/s. The underlying hardware is
an Intel Xeon E5-2676 v3 clocked @ 2.4 GHz with 30MB of CPU cache
\cite{aws_instances}, running Ubuntu 14.04 with kernel version
4.4.0-34-generic. We used \Apache~Benchmark to generate 5000 HTTPS requests
across 10 concurrent connections and measure the performance of the server. We
consider the following four scenarios:
\begin{enumerate}
  \item Using a cipher lacking support for forward secrecy, RSA-AES256-SHA,
    with the design where session keys reside in the untrusted component
  \item Using a cipher lacking support for forward secrecy,
    RSA-AES256-GCM-SHA384, with the design where session keys reside in the
    trusted component
  \item Using a cipher offering support for forward secrecy,
    ECDHE-AES256-GCM-SHA384, with the design where session keys reside in the
    untrusted component
  \item Using a cipher offering support for forward secrecy,
    ECDHE-AES256-GCM-SHA384, with the design where session keys reside in
    the trusted component
\end{enumerate}

For each of the above scenarios we run the test for a variable number of
busywait cycles (from 0 to 50,000) and response sizes (from 1KB to 4MB). We
setup NGINX with a single worker process due to the limitations mentioned in
Section \ref{sec:implementation}.

\subsubsection*{Results}
Figure \ref{fig:cycles-requests} displays the server responses/s for varying
number of busywait cycles. Figure \ref{fig:sizreqs} demonstrates the server
responses/s for variable response size and zero number of busywait cycles.
Note the logscale on the x axis.
 \begin{figure}[H]
   \centering
   \begin{tikzpicture}
     \begin{axis}[pretty,ylabel = Responses/Second, ymin = 0, xlabel =
       Number Of Busywait Cycles (Cycles * $10^4$), legend pos=outer
       north east, xtick scale label code/.code = {}]
       \addplot table[y=Key Outside RSA, mark = *] from \cyclereqs;
       \addlegendentry{Session Keys Outside \textit{RSA}}; \addplot
       table[y=Stock RSA, mark = *] from \cyclereqs;
       \addlegendentry{Stock \textit{RSA}}; \addplot
       table[y=Stock ECDHE, mark = *] from \cyclereqs;
       \addlegendentry{Stock \textit{ECDHE}}; \addplot
       table[y=Key Outside ECDHE, mark = *] from \cyclereqs;
       \addlegendentry{Session Keys Outside \textit{ECDHE}}; \addplot
       table[y=Key Inside RSA, mark = *] from \cyclereqs;
       \addlegendentry{Session Keys Inside\textit{RSA}}; \addplot
       table[y=Key Inside ECDHE, mark = *] from \cyclereqs;
       \addlegendentry{Session Keys Inside \textit{ECDHE}};
     \end{axis}
   \end{tikzpicture}
   \caption{Graph of busywait cycles vs. requests per second}
   \label{fig:cycles-requests}
 \end{figure}

 \begin{figure}[H]
   \centering
   \begin{tikzpicture}
     \begin{axis}[pretty,ylabel = Responses/Second, ymin = 0, xmode =
       log,xlabel = Response Size (KB), legend pos=outer north east,
       xtick scale label code/.code = {}]
       \addplot table[y=Stock RSA, mark = *] from \sizereqs;
       \addlegendentry{Stock \textit{RSA}}; \addplot table[y=Stock
       ECDHE, mark = *] from \sizereqs; \addlegendentry{Stock
         \textit{ECDHE}}; \addplot table[y=Key Outside RSA, mark = *]
       from \sizereqs; \addlegendentry{Session Keys Outside
         \textit{RSA}}; \addplot table[y=Key Outside ECDHE, mark = *]
       from \sizereqs; \addlegendentry{Session Keys Outside
         \textit{ECDHE}};
     \end{axis}
   \end{tikzpicture}
   \caption{Graph of response size vs. requests per second}
   \label{fig:sizreqs}
 \end{figure}

\subsubsection*{Analysis}
As discussed in Section \ref{sec:cpu-instr-analysis}, note that there
occur 13 context switches for a request in the case for RSA and 21
context switches for each request in the case of ECDHE. We expected
the performance penalty to be around 15\%, because each context switch
causes a TLB flush, which in turn results in slower execution until
the TLB is repopulated. Figure \ref{fig:cycles-requests} validates our
hypothesis. It shows a performance penalty of \textasciitilde18\% for
RSA ciphers and \textasciitilde33\% for ECDHE ciphers.

% TODO: document keys inside/outside for the first graph
% TODO: document the second graph
\end{document}

%%% Local Variables:
%%% mode: latex
%%% TeX-master: "../../../main"
%%% End:
