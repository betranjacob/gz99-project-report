\documentclass[../../../main.tex]{subfiles}

\begin{document}
\label{sec:perfmodel}
To model the performance of our prototype we implemented a second
version of our enclave program that has no dependencies on OpenSGX and
replaces certain SGX instructions with busy waits, simulating the
overhead in executing them. In doing so, we make the same assumptions
as~\cite{Baumann14}, namely:
\begin{enumerate}
  \item We assume that the CPU used in testing performs the same as an
    SGX-enabled CPU for all non-SGX instructions
  \item We assume that the EPC is large enough to accommodate the
    entirety of the trusted component and any data structures created
    after program start-up
\end{enumerate}
leaving the overhead of SGX instructions, memory encryption, and
asynchronous exits unaccounted for.

SGX instructions that bootstrap the enclave and verify its integrity
are executed once, at startup, and therefore have no effect on runtime
performance. As a result, we only simulate the overhead of
instructions used to switch to and from the \textit{enclave program},
EENTER, ERESUME and EEXIT. Simulating the slow down in memory accesses
due to the need for decrypting RAM contents before being able to
access them in processors cache, was carried out by reducing the
memory's clock in~\cite{Baumann14}. Such a proxy works for their
system because it executes within an enclave in its entirety. Clearly,
such an approach would underestimate the performance of our system
because only the trusted component incurs the overhead of memory
encryption. This issue is addressed in Section~\ref{sec:mem-analysis}.

\end{document}

%%% Local Variables:
%%% mode: latex
%%% TeX-master: "../../../main"
%%% End:
