\documentclass[../../../main.tex]{subfiles}

\begin{document}
\label{sec:perfmodel}
To model the performance of our prototype we implemented a second version of
the \enclaveprogram, that we call \busywait~or \enclavemodel, which
has no dependencies on OpenSGX. For our model, we make the same assumptions
as~\cite{Baumann14}, namely we assume that:
\begin{enumerate}
  \item The CPU used in testing performs the same as an SGX-enabled CPU for
    all non-SGX instructions.
  \item The EPC is large enough to accommodate the entirety of the trusted
    component and any data structures created after program start-up.
\end{enumerate}

The \busywait~replaces certain SGX instructions with busy waits (by looping on
the RDTSC register), simulating the overhead in executing them. SGX
instructions that bootstrap the enclave and verify its integrity are executed
only once, at startup, and therefore have no effect on runtime performance. As
a result, we only simulate the overhead of instructions used to switch to and
from the \enclaveprogram, i.e.\ EENTER, ERESUME and EEXIT.

Simulating the slow down in memory accesses due to the need for decrypting
DRAM contents before being able to access them in processors cache was carried
out by reducing the memory's clock in~\cite{Baumann14}. Such a proxy works for
their system because it executes within an enclave in its entirety. Clearly,
this approach would underestimate the performance of our system because only
the trusted component incurs the overhead of memory encryption. This issue is
addressed in Section~\ref{sec:mem-analysis}.

We leave the overhead incurred by asynchronous enclave exits unaccounted for.
\end{document}

%%% Local Variables:
%%% mode: latex
%%% TeX-master: "../../../main"
%%% End:
