\documentclass[../../../main.tex]{subfiles}

\begin{document}
\label{sec:tcb-analysis}
To estimate the size of the TCB of the \enclaveprogram~after our partitioning,
we generated a dynamic call graph (using callgrind) under one RSA and one ECDHE
request, separately for session keys inside and outside the enclave as well as
stoc NGINX + LibreSSL. We then crawled the source files in search for lines
between the function names obtained from callgrind and ``\}'' using
\textit{sed} and \textit{grep}. We excluded syscalls, libc functions,
functions defined in multiple files and functions which names are
declared with preprocessor macros. The resulting list is not fully accurate
but rather gives a rough estimate. Rest of the functions were copied into a
separate C file for each of the designs and the resulting SLOC was counted
with cloc. The results are presented in Table~\ref{tab:tcb-estimate}.

Using cloc, we also compared the modified source code of LibreSSL and NGINX to
their stock versions. We report these results in Table~\ref{tab:libressl-diff}
and Table~\ref{tab:nginx-diff}.

\subsubsection*{Results}
The columns in tables \ref{tab:libressl-diff} and~\ref{tab:nginx-diff} have
the following meaning:
\begin{itemize}
  \item \textit{\footnotesize == files}: files that have not been changed;
  \item \textit{\footnotesize != files}: modified files;
  \item \textit{\footnotesize + files}: new files;
  \item \textit{\footnotesize == code}: lines of code that have not been
    changed;
  \item \textit{\footnotesize != code}:  modified lines of code;
  \item \textit{\footnotesize + code}: new lines od code;
  \item \textit{\footnotesize - code}: removed lines of code;
\end{itemize}

\begin{table}[H]
  \center
  \footnotesize
  \pgfplotstabletypeset[assign column
    name/.style={/pgfplots/table/column name={\textbf{#1}}},col
    sep=comma, header=has colnames]{results/tcb_analysis.csv}
  \caption{TCB size estimation based on the SLOC}
  \label{tab:tcb-estimate}
\end{table}

\begin{table}[H]
  \resizebox{\linewidth}{!}{
  \pgfplotstabletypeset[
    /pgfplots/table/columns={Language, == files, != files, + files, == code, != code, + code, - code},
    assign column name/.style={/pgfplots/table/column name={\textbf{#1}}},
    col sep=comma,
    header=has colnames
  ]
  {results/libressl_sloc_diff.csv}}
  \caption{Changes resulting from LibreSSL instrumentation}
  \label{tab:libressl-diff}
\end{table}

\begin{table}[H]
  \resizebox{\linewidth}{!}{
  \pgfplotstabletypeset[
    /pgfplots/table/columns={Language, == files, != files, + files, == code, != code, + code, - code},
    assign column name/.style={/pgfplots/table/column name={\textbf{#1}}},
    col sep=comma,
    header=has colnames
  ]
  {results/nginx_sloc_diff.csv}}
  \caption{Changes resulting from NGINX instrumentation}
  \label{tab:nginx-diff}
\end{table}

\subsubsection*{Analysis of Results}
The resulting TCB of our implementation is over 60\% smaller compared to stock
NGINX based on the lines of code required to handle ECDHE and RSA ciphers.
However, we might have missed some code paths since the callgraphs were
generated dynamically and we only exercised successful requests. Moreover, our
system assumes the CPU and Intel attestation process to be part of the TCB
which is not reflected in the presented tables.

Our implementation required instrumentation of 8 LibreSSL source files. We also
added one file (SGXBridge) to handle the communication between trusted and
untrusted components. To achieve our goal we added less then 600 LOC, which
increased the LibreSSL code base by 0.3\%. The changes to NGINX were
negligible. Our \enclaveprogram~is 566 LOC.

\end{document}

%%% Local Variables:
%%% mode: latex
%%% TeX-master: "../../../main"
%%% End:
