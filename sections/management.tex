\documentclass[../main.tex]{subfiles}
\begin{document}

We have spent the initial two weeks of the project researching what the exact
problem we try to tackle is and what would be the best approach to solve it.
We looked at available opensource software that we could use as a starting
point, focusing on documentation available and taking into account its current
and prijected market share. During that time we have also explored other
research in the area and its approach to performance evaluation.

During that time we were all trying to get a better insight into the scope of
the project and the software landscape and therefore we did not split the work
between individuals.

We came to the conclusion that support for hardware SGX, including BIOS and
Operating System, as well as SDK from intel is very limited or lacking
documentation and, hence, decided to use a software emulator OpenSGX instead,
which required us to be more careful in reasoning about our future prototype
performance in real life scenario.

We also decided to use LibreSSL as oposed to the more mainsream OpenSSL as the
former seemed to be more concerned about limiting the size of its TCB. We also
decided to use NGINX instead of Apache as the former has more modern design,
better resource management and steadily gains popularity.

Having these in mind, we initially divided our group into three subgroups:
\begin{itemize}
	\item Ahmed and John working on the SGX part
	\item Betran and Wiktor working on LibreSSL and NGINX
	\item Chiraphat working on the tests setup and performance evaluation
\end{itemize}

He have identified the major and minor goals and tasks as well as milestones
we need to reach for the project to succeed. These are presented in the gant
chart in Figure~\ref{fig:gantchart} below.
% TODO: some more details here?

\begin{figure}[H]
  \centering
  \includegraphics[scale=0.25]{images/gant.png}
  \caption{Project timeline}
  \label{fig:gantchart}
\end{figure}

As we only had ten weeks left to complete the project we had to employ
a management technique that would allow to rapidly adapt to any delays and
minimize the risks. For that reason we heavily relied on constant
communication with the group members as well as on weekly meetings to share
progress with our supervisor Brad, assess whether we need to shift manpower to
any blocking tasks. Additionally, we held our own meeting to internally
discuss the insights and hints from Brad and our own, make sure we agree in the
understanding of the current state of the project, and plan and distribute
work for the next week.\\

We have used Slack as our daily communication channel since we had it already
set up for other NCS group works. We have setup github repositories for code
and report as we decided to use LaTex for ease of collaboration and to follow
research best practices.\\

Over the course of the project some tasks turned out to take less time than we
planned but we also encountered a few bottlenecks and implementation
difficulties that required assignment of more team members. In particular we
managed to acomplish most of our initial goal of isolating the private key for
RSA handshake and performing the necessary remote attestation fairly quickly.
We could then shift the effort to support ciphers offering forward security
and compartmentalization of session keys. The work was roughly split as
follows.

Ahmed fixed OpenSGX remote attestation and provinsioning mechanism, he
developed SGX expertise that was helping us with correct design decisions and
identification of SGX performance hits. He was also the mastermind behind
report structure and editorial work.

John helped with initial OpenSGX setup and testing, he then moved to help with
the libressl compartmentization, i.e. computation of key block inside the
enclave and encrypt / decrypt oracle interface. He also helpe with AWS setup
and build scripts.

Betran was responsible for initial NGINX and LibreSSL exploration. He then
added support for forward secrecy ciphers and carried initial work on
encrypt / decrypt interface.

Wiktor prepared the initial build environment and callgate interface between
untrusted and trusted code. He also helped with the setup of the test environment and integration between different versions of our programs.

Chiraphet developed the performance evaluation model and tested our implementation on various machines. He was responsible for the choice of performance profiling tools and finally, executing the tests and gathering results.

\end{document}
