\documentclass[../main.tex]{subfiles}
\begin{document}

% - Outline of how the work was split up.
% - How collaborative work was planned.
% - Who did what.
% - How agreement was reached.
% - How essential project information was managed and communicated.
% - How integration and testing was organised.

We spent the initial two weeks of the project in researching about the problem
definition and what would be the best approach to solve it. We looked at the
available open source softwares that we could use as a starting point, focusing
on documentation available and taking into account its current and projected
market share. During that time, we also explored relevant research in the area
and studied approaches we could use for performance evaluation. These tasks
mostly involved collaborative work, trying to get a better insight into the
scope of the project and the software landscape and therefore we did not split
the work between individuals.

We came to the conclusion that the support for the hardware SGX, including BIOS and
Operating System, as well as the \Intel~SDK is very limited or lacking
documentation; hence, we decided to use the aforementioned emulator OpenSGX
instead. This required us to be more careful in reasoning about our future
prototype performance in a real production environment.

As discussed in Section \ref{sec:implementation}, we decided to use NGINX and
LibreSSL to implement a prototype of our design. Having these in
mind, we initially divided our group into three subgroups:

\begin{itemize}
	\item Ahmed and John working on the SGX part.
	\item Betran and Wiktor working on LibreSSL and NGINX.
	\item Chiraphat working on the tests setup and performance evaluation.
          He has also identified the major and minor goals, tasks and
          milestones we needed to reach for the project to succeed. %These are
          %presented in the gant chart in Figure~\ref{fig:gantchart} below.
\end{itemize}

% TODO: some more details here?

%\begin{figure}[H]
%  \centering
%%  \includegraphics[scale=0.25]{images/gant.png}
%  \caption{Project timeline}
%  \label{fig:gantchart}
%\end{figure}

Taking into account the available time span of ten weeks, we had to employ a
management technique that would allow us to rapidly adapt to any delays and
minimize the risks. We heavily relied on constant communication with the group
members as well as on the meetings with our supervisor, Professor B. Karp, which
played a critical role in ensuring our progress. We held these meetings in a
weekly basis where we would share our progress, discuss design or technical
issues and present our plans for the following weeks. These meetings also gave
us the chance to assess whether to shift manpower to any blocking tasks.
Additionally, we held our own meetings to internally discuss the insights and
hints obtained through the discussions with our supervisor. This ensured we agreed
on the understanding of the current state of the project, plan and distribute work
for the upcoming week.

We used Slack as our daily and synchronous communication channel since it
provides several facilities such as file sharing, integration with popular
version control services such as GitHub, notification system \&c. We used
Git as a version control system (both for code and report writing) and GitHub
as a service to store our repositories upstream. We decided to use \LaTeX~for
the ease of collaboration and to follow the research’s best practices.

Over the course of the project some tasks turned out to take less time than we
planned. We also encountered a few bottlenecks and implementation
difficulties that required assignment of more team members. In particular, we
managed to accomplish our initial goal in isolating the private key for the RSA
handshake and performing the necessary remote attestation sooner than we had
initially planned. We could then shift the effort to support ciphers offering
forward security and compartmentalization of session keys. The work was roughly
split as follows:

Ahmed fixed OpenSGX remote attestation and provisioning mechanism, he
developed SGX expertise that was helping us with correct design decisions and
identification of SGX performance hits. He was also the mastermind behind
report structure and editorial work.

Ioannis initially helped in setting up OpenSGX and then focused on LibreSSL
compartmentalization. He isolated the computation of session keyblock inside
the enclave, implemented the encrypt/decrypt oracle interface and added support
for concurrent connections and SSL session resumption within the enclave. He
also worked on the build scripts and setting up AWS.

Betran was responsible for initial NGINX and LibreSSL exploration. He then
added support for forward secrecy ciphers and carried initial work on the
encrypt/decrypt interface. Also contributed to various sections in the report. 

Wiktor prepared the initial build environment and developed call gate interface
between untrusted and trusted code for the RSA ciphers. He also helped with
the setup of the test environment and integration between different versions
of our programs.

Chiraphat developed the performance evaluation model and tested our
implementation on various machines. He was responsible for the choice of
performance profiling tools and finally, executing the tests and gathering
results.
\end{document}

%%% Local Variables:
%%% mode: latex
%%% TeX-master: "../main"
%%% End:
