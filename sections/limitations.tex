\documentclass[../main.tex]{subfiles}

\begin{document}
\label{sec:limitations}
Our design does not provide protection against Denial of Service
attacks and we consider this attack vector as being out of scope. We
were initially concerned that our use of blocking I/O operations on
the named pipe interface exposes a DoS vulnerability. In case the
non-enclave process gets compromised, an adversary can launch a DoS attack
by writing arbitrary bytes in the named pipe used for interprocess
communication. This would cause the enclave or the non-enclave process to
either block on read(), if less than the expected bytes exist in the
pipe; or exit due to a logical error introduced by the arbitrary bytes
read. A more sophisticated implementation of the named pipe interface
would be to use asynchronous I/O operations with timeouts as signals
to proceed with serving the next request. This would only make our
design a bit more resilient to such attacks, though still not
invulnerable. The threat model we are considering implies that
everything running outside the enclave can be potentially compromised. Thus,
an adversary that takes control of the hypervisor or the host OS, has
way more trivial choices on how to launch a DoS attack such as
blocking the NIC from incoming packets or even shutting down the
machine itself, than writing an exploit destined for our named pipe
interface. It is clear, our compartmentalisation of the openssl
library does not provide any additional conveniences for the attacker
to launch a DoS attack. Also, a DoS attack only constitutes a liveness
threat, rather a privacy threat, which fully complies with SGX's
native goal of providing confidentiality and integrity guarantees.
\end{document}
