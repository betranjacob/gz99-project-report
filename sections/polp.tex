\documentclass[../main.tex]{subfiles}

\begin{document}
The first of our goals (see Section~\ref{sec:introduction}) has been
extensively addressed in previous work ~\cite{Bittau08, Krohn2004}
through use of the principle of least privilege. The principle of
least privilege (PoLP) requires that components of a system operate
with the minimum resources required to complete their respective
tasks. PoLP is heavily utilized in systems security research to design
systems that maintain their integrity in the face of an adversary
capable of exploiting some of their components. In contrast, the most
popular web servers today, including Apache and NGINX, execute as
monolithic applications where all of their processes have the same
level of privilege and access to the sensitive key material.
Exploiting any one of these processes may, therefore, lead to
compromising the long-term private key.

To motivate the design of our system, we begin by discussing a system
called Wedge~\cite{Bittau08} that was used to refactor Apache, by
enforcing PoLP, securing the private key in the face of an adversary
who is capable of exploiting the web server application. However, the
design introduced in the Wedge paper assumed a trusted OS, hypervisor,
and cloud-provider. Moreover, Wedge's authors concentrated their
efforts on SSL/TLS cipher suites that do not offer forward secrecy,
and the scenario where only the client verifies the server's identity
as part of the SSL/TLS handshake. Figure~\ref{fig:pristrsashake}
illustrates the resulting SSL/TLS handshake. We refer to this
handshake as the ``RSA'' handshake for brevity\footnote{The asymmetric
  cipher usually used in this handshake is RSA; hence the name}. The
asymmetric key exchange step is not necessary here, as mentioned in
the previous section. The symmetric key negotiation step comprises of
the client generating a random value for \texttt{PremasterSecret},
encrypting it in the server's public key, and sending the resulting
payload to the server. Observe that the long-term private key, as a
result, is only used in decrypting \premaster.

\begin{figure}[H]
  \centering
  \includegraphics[scale=0.4]{images/rsa-handshake-pristine.pdf}
  \caption[SSL/TLS ``RSA'' handshake]{SSL/TLS handshake for ciphers
    that do not offer forward secrecy. $K$ here is the server's public
    key, and the braces around \texttt{PremasterSecret} indicate the
    payload is encrypted}
  \label{fig:pristrsashake}
\end{figure}

Moving forward, we list the attacks possible against a monolithic web
server, and then detail how the design implemented with Wedge resolved
these vulnerabilities. In this section we only consider a threat
model where the adversary is capable of passively eavesdropping on
secure communication channels and exploiting unprivileged components
of the server. We did not consider the second threat model detailed in
the Wedge paper, wherein an attacker is capable of actively modifying
payloads exchanged between the web server and client, because the
solution presented there does not hold if we remove the assumption
that the OS and cloud provider are trusted. This is a consequence of
the web server's user-level processes (dynamic content generation
scripts, databases \&c.)  requiring data in plain-text to complete
their tasks. As such, even if we secure the encrypt/decrypt interface
using the solution in the Wedge paper, plain-text data would
be available in a non-privileged process's memory. This is further
discussed in Section~\ref{sec:design}.

\subsubsection*{Possible Attacks}

There are two main attacks that may be mounted by an adversary capable
of exploiting \textit{only} unprivileged components of the web
server application and passively eavesdropping on payloads exchanged
between the server and the client:
 
\begin{enumerate}
  \item The adversary could leak the private key from the network
    facing worker process. This could be done by reading the private key
    from disk directly or from the process's memory space. The attack is
    illustrated in Figure~\ref{fig:attack1}.

    \begin{figure}[H] \centering
      \includegraphics[scale=0.15]{images/attack1.png}
      \caption{Exploiting the network facing component to leak the
        private key}
      \label{fig:attack1}
    \end{figure}

  \item The adversary may record traffic previously exchanged over the SSL/TLS
    channel and then exploit a naive session key generation interface to
    acquire the session keys generated in that prior exchange. A naive interface is
    one that accepts \crandom, \srandom, and \premaster~ where $K$ is the
    web server's public key. Such an interface allows an adversary, after
    exploiting the unprivileged component, to generate any previously
    eavesdropped session's symmetric key and is, therefore, no different
    to having read access to the private key from the adversary's
    perspective. This exploit only works if the cipher suite used does not offer
    forward secrecy, and it only affects the single session that was
    eavesdropped. The attack is illustrated in Figure~\ref{fig:attack2}.

    \begin{figure}[H] \centering
      \includegraphics[scale=0.15]{images/attack2.png}
      \caption{Exploiting the network facing component and the naive
        session key generation interface to generate session keys for
        eavesdropped session}
      \label{fig:attack2}
    \end{figure}
\end{enumerate}

\subsubsection*{Proposed Solution}

The solution proposed in the Wedge paper is achieved through
partitioning the session key generation code into its own logical
compartment \footnote{This compartment is called an sthread in Wedge's
  nomenclature} that executes at high privilege. This partitioning can
be seen in Figure~\ref{fig:wedge-partition}.

\begin{figure}[H]
  \centering
  \includegraphics[scale=0.25]{images/compartment_01.png}
  \caption{Partitioning Scheme to protect against leaking the private
    key. The ovals represent system components, shaded ones are
    privileged. Dashed arrows from one component to another indicate a
    component's abillity to invoke a privileged component's interface.
    Boxes correspond to memory objects. The double-headed arrows
    between a system component and a memory object indicates a
    component's read/write privileges to the memory object. A single
    headed arrow corresponds to read-only access.~\cite{Bittau08}}
  \label{fig:wedge-partition}
\end{figure}

The low privilege worker process does not possess the private key, but
must rely on an interface exposed by the session key generation
process. The interface takes \crandom~ and \premaster~ as arguments,
both of which are provided by the client, while the high privilege
component generates a fresh \srandom~ upon invoking the interface. The
freshness property ensures that, even if an adversary exploits the
worker process, previous session keys cannot be obtained by simply
invoking the session key generation interface.

While this design secures the private key against the previously
described adversary, it is not, by itself, sufficient to secure the
key against an adversary capable of exploiting the OS, or from a
malicious cloud-provider with full access to the physical machine.
Such adversaries may leak the key from the privileged component's
memory space by exploiting the OS, or, in the case of the malicious
cloud provider, simply read the key from disk\footnote{Even if the
  long-term private key is encrypted on disk, the key for decrypting
  it must be present in the binary file of the privileged component
  (which is unencrypted), or in the process's memory at runtime in
  plain-text}. Other prior works has used a hardware component called
a trusted platform module in an effort to secure the private key
against the adversaries described above.

\end{document}

%%% Local Variables:
%%% mode: latex
%%% TeX-master: "../main"
%%% End:
