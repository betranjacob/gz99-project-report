\documentclass[../main.tex]{subfiles}

\begin{document}
The first of our goals has been extensively addressed in previous
works ~\cite{Bittau08, Krohn2004} through use of the principle of
least privilege. The principle of least privilege (PoLP) requires
that components of a system operate with the minimum resources
required to complete their respective tasks. PoLP is heavily utilized
in systems security research to design systems that maintain their
integrity in the face of an adversary capable of exploiting some of
their components. In contrast, the most popular web servers today
execute as monolithic applications, including Apache and NGINX, where
all of their processes have the same level of privilege and have
access to the sensitive key material. Exploiting any one of these
processes may therefore lead to leaking the private key.

To motivate the design of our system, we begin by discussing a system
called Wedge~\cite{Bittau08} that was used to refactor Apache, by
enforcing PoLP, securing the private key in the face of an adversary
who is capable of exploiting the web server application. First, we
list the attacks possible against a monolithic web server and then
detail how the design implemented with Wedge resolved these
vulnerabilities.

We only consider a threat model where the adversary is capable of
passively eavesdropping on secure communication channels, and
exploiting unprivileged components of the server. We did not consider
the second threat model detailed in the Wedge paper, wherein an
attacker is capable of actively modifying packets exchanged between
the web server and client, because the solution presented in there
does not hold if we remove the assumption that the OS and cloud
provider are trusted. A consequence of the web server's user-level
processes (dynamic content generation scripts, databases \&c.)
requiring data in plain-text to complete their tasks, even if we secure
the encrypt/decrypt interface using the solution in the Wedge paper,
is that plain-text data would be available in the non-privileged process's
memory. This is further discussed in Section~\ref{sec:design}.

\subsubsection*{Possible Attacks}

There are two main attacks that may be mounted by an adversary capable
of exploiting \textit{only} the network facing component of the web
server application and passively eavesdropping on packets exchanged
between the server and the client:

\begin{enumerate}
  \item The adversary could leak the private key from the network facing
    component, which has to run as root to bind to port 80.  This could
    be by reading the private key from disk directly or from the
    process's memory space. The attack is illustrated in
    Figure~\ref{fig:attack1}.

	\begin{figure}[H]
          \centering
          \includegraphics[scale=0.15]{images/attack1.png}
          \caption{Exploiting the network facing component to leak the
            private key}
          \label{fig:attack1}
	\end{figure}

  \item The adversary may record traffic exchanged over the
    SSL/TLS channel and then exploit a naive session key
    generation interface to acquire the session keys used in that
    exchange. A naive interface is one that accepts \crandom,
    \srandom, and \premaster~ where $K$ is the web server's public
    key.  Such an interface allows an adversary, after exploiting
    the unprivileged component, to generate any previously
    eavesdropped session's symmetric key (assuming a cipher that
    does not provide the property of perfect forward secrecy) and
    is, therefore, no different to having read access to the
    private key from the adversary's perspective. This exploit
    only works if the cipher used is one that does not offer
    perfect forward secrecy (PFS), and it only affects the single
    session which was eavesdropped. The attack is illustrated in
    Figure~\ref{fig:attack2}.

	\begin{figure}[H]
          \centering
          \includegraphics[scale=0.15]{images/attack2.png}
          \caption{Exploiting the network facing component and the
            naive session key generation interface to generate session
            keys for eavesdropped session}
          \label{fig:attack2}
	\end{figure}
\end{enumerate}

\subsubsection*{Proposed Solution}

The solution proposed in the Wedge paper is achieved through
partitioning the session key generation code into its own logical
compartment \footnote{This compartment is called an sthread in Wedge's
  nomenclature} that executes at high privilege. This partitioning can
be seen in Figure~\ref{fig:wedge-partition}.

\begin{figure}[H]
  \centering
  \includegraphics[scale=0.25]{images/compartment_01.png}
  \caption{Partitioning Scheme to protect against leaking the private
    key~\cite{Bittau08}}
  \label{fig:wedge-partition}
\end{figure}

The low privilege worker process does not have possess to the private
key, but must rely on an interface to the session key generation
process. The interface takes \crandom~ and \premaster~ as arguments,
both of which are provided by the client, while the high privilege
component generates a fresh \srandom~ upon invoking the interface. The
freshness property ensures that, even if an adversary exploits the
worker process, previous session keys cannot be generated by simply
invoking the session key generation interface.

While this design secures the private key against the previously
described adversary, it is not, by itself, sufficient to secure the
key against an adversary capable of exploiting the OS, or from a
malicious cloud-provider with full access to the physical
machine. Such adversaries may leak the key from the privileged
component's memory space by exploiting the OS, or, in the case of the
malicious cloud provider, simply read the key from disk\footnote{Even
  if the key is encrypted on disk, the decryption key must be present
  in the plain-text binary file for the privileged component or in the
  process's memory, at runtime, in plain-text}.  A hardware component,
called a trusted platform module, has been utilised in an effort to
secure the private key against the adversaries described above.

\end{document}
