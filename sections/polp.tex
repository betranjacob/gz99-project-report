\documentclass[../main.tex]{subfiles}

\begin{document}     
\subsection{The Principle Of Least Privilege}     

As previously stated the principle of least privilege (PoLP) requires that
components of a system operate with the minimum resources required to complete
their respective tasks. PoLP is heavily utilized in systems security research to
design systems that maintain their integrity in the face of an adversary capable
of exploiting some of their components. In contrast, the most popular web
servers today execute as monolithic applications, including Apache and NGINX,
where all of their processes have the same level of privilege and have access to
the sensitive key material. Exploiting any one of these processes may therefore
lead to leaking the private key.

To motivate the design of our system, we begin by discussing a system called
Wedge~\cite{Bittau08} that was used to refactor Apache, by enforcing PoLP,
securing the private key in the face of an adversary who is capable of
exploiting the web server application. First, we list the attacks possible
against a monolithic web server and then detail how the design implemented
with Wedge resolved these vulnerabilities. 

We only consider a threat model where the adversary is capable of passively
eavesdropping on secure communication channels. We did not consider the second
threat model, detailed in the Wedge paper, where an attacker is capable of
actively modifying packets in flight between the web server and client because
the solution presented in the Wedge paper does not hold if we remove the
assumption that the OS and cloud provider are trusted as the web server's user-
level processes (dynamic content generation scripts, databases \&c.) need to
parse web request data in the clear, and as a consequence, even if we secure the
encrypt/decrypt interface using the solution in the Wedge paper, the data would
be available in the non-privileged process's memory. This is further discussed
in  Section~\ref{sec:design}.

\subsubsection*{Possible Attacks}

There are two main attacks that may be mounted by an adversary capable of
exploiting \textit{only} the network facing component of the web server
application and passively eavesdropping on packets exchanged between the server
and the client:

\begin{enumerate}     
	\item The adversary could leak the private key from
	the network facing component, which has to run as root to bind to port 80.
	This could be by reading the private key from disk directly or from the
	process's memory space. The attack is illustrated in
	Figure~\ref{fig:attack1}.

	\begin{figure}[H]
		\centering
		\includegraphics[scale=0.15]{images/attack1.png}
		\caption{Exploiting the network facing component to leak the
		private key}
		\label{fig:attack1} 
	\end{figure}

	%add a foot note explaining the naive session key generation?
	\item The adversary may record traffic exchanged over the SSL/TLS channel
	and then exploit a naive session key generation interface to acquire the
	session keys used in that exchange. This exploit only works if the cipher
	used is one that does not offer perfect forward secrecy (PFS). The session
	key generation operations require use of the private key, but the adversary
	does not need to learn the key to complete this attack. However, th attack
	only affects the single session which was eavesdropped. The attack is
	illustrated in Figure~\ref{fig:attack2}.

	\begin{figure}[H]
		\centering
		\includegraphics[scale=0.15]{images/attack2.png}
		\caption{Exploiting the network facing component and the naive 
		session key generation interface to generate session keys for
		eavesdropped session}
		\label{fig:attack2}
	\end{figure}
\end{enumerate}

\subsubsection*{Proposed Solution} 
The solution proposed in the Wedge paper is through partitioning the session key
generation code into its own logical compartment \footnote{This compartment
called an sthread in Wedge's nomenclature} that executes at a high privilege
level. This partitioning can be seen in Figure~\ref{fig:wedge-partition}.

\begin{figure}[H]
	\centering
	\includegraphics[scale=0.25]{images/compartment_01.png}
	\caption{Partitioning Scheme to protect against leaking the private 
	key~\cite{Bittau08}}
	\label{fig:wedge-partition}
\end{figure}

\end{document}