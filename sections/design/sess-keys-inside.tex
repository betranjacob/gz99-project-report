\documentclass[../../main.tex]{subfiles}

\begin{document}
In both designs described above, \texttt{MasterSecret} is returned to
the untrusted component. As previously mentioned, the
\texttt{MasterSecret} is used to compute the session key block; hence,
an adversary, after exploiting the untrusted component, may leak the
session key block, or leak \texttt{MasterSecret} and derive the
session key block themselves. Consequentially, an attacker may choose
to store session key blocks and eavesdropped sessions, and decrypt the
stored sessions at his leisure. Yet, requests received from the
client have to be in plain-text for processing by unprivileged user
programs. This section describes an augmentation to the designs
presented above, which attempts to resolve the highlighted issue, and
explains the resulting security properties.

A possible solution to the outlined problem would be to implement the
user programs that require plain-text as part of the \textit{enclave
  program} and, by extension, the TCB. As a result, session keys may
reside within the enclave, and, aside from packets that are plain-text
in the SSL/TLS handshake, all output from the \textit{enclave program}
to the unpriveleged/untrusted network facing component would be
encrypted. This design, however, introduces a large TCB that
incorporates all of the user-written code, making it difficult to
reason about its security.

Alternatively, we expanded upon the designs explained earlier as
follows: instead of returning the \texttt{MasterSecret} to the
untrusted component, we compute the session key block, and store it
within the enclave. We then provide a interface that allows the
untrusted component to decrypt data received from a client and an
interface to encrypt data outgoing to a client. Figure
~\ref{fig:keys-inside} illustrates this design.

\begin{figure}[H]
  \centering
  
  \caption{SSL/TLS handshake with session keys in the enclave}
  \label{fig:keys-inside}
\end{figure}

% TODO: EXPLAIN DESIGN
% TODO: EXPLAIN DESIGN'S PROPERTIES
The server asks the enclave to generate the session key block using
the pre-negotiated secrets and this key block does never leave the
trusted component. At this point, the client signals the server with a
\texttt{ChangeCipherSpec} message about transitioning to a specific
CipherSpec and the server interacts with the enclave to initialize the
client's cipher specific environment. The server now prepares for
receiving the \texttt{ClientFinished} message by calculating the
digest of the handshake messages it received so far \footnote{As
  defined by TLS 1.2, the digest enclosed in the \texttt{Finished}
  messages does not include the \texttt{Finished} messages
  themselves.}. This step entails a dependency of the established
session key, thus a context switch to the trusted environment is
required for calculating the digest of the handshake buffer. Upon
receiving the encrypted \texttt{ClientFinished} message, the server
forwards the message intact to the enclave for decryption through the
decrypt interface and calculates the MAC of the decrypted message to
verify its authenticity and integrity. If this process successfully
completes, the server signals the client for transitioning to a
specific cipher and sets up the appropriate cipher environment in the
enclave as well. Finally, it asks the enclave to calculate the digest
of the handshake messages it received so far and encrypts this digest
within the encalve using the established session keys. If the client
successfully verifies all security properties of the
\texttt{ServerFinished} message, the SSL connection is considered
established. Note that, as the protocol defines, the interface used
for encryption/ decryption of the \texttt{Finished} messages is the
same used for encryption/ decryption of the user data travelling over
the network. Thus, every subsequent data record will be
encrypted/decrypted by the same interface used for
encrypting/decrypting the \texttt{Finished} messages, with the session
keys never being exposed to the untrusted component.


\end{document}

%%% Local Variables:
%%% mode: latex
%%% TeX-master: "../../main"
%%% End: