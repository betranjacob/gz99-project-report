\documentclass[../../main.tex]{subfiles}

\begin{document}

\paragraph{Forward Secrecy and ECDHE Handshake}
In RSA key exchange process, all the messages exchanged between server
and client can be compromised if the server’s private key is ever
disclosed. In specific, an active attacker who has recorded all the
data communication between the server and client for a certain period
of time and stores it until such a time he has access to the private
key. The adversary can now decrypt all the historic data using the
compromised private key. This is possible because the attacker will
have access to all the seeds ( Client Random, Server random and
Pre-Master secret ) required to generate the symmetric session keys
This capability to decode historic data at any point of time in the
future pose a serious threat to the secure communication over the cyberspace.\\

\noindent
Forward secrecy is a feature in the key arrangement protocol, which
guarantees that the symmetric encryption keys can not be compromised
even if the long term key ( server private key ) is compromised. To
achieve forward secrecy we employed elliptic curve
diffie-hellman(EC-DHE) key exchange mechanism. Most of the web servers
prefer elliptic curves variant over standard DHE as it yields better
performance with lesser number of key bits[1]. DHE handshake uses
private key only to authenticate the server and generates fresh set of
symmetric keys for every new session. The keys are computed from
random EC parameters, which is never reused and also deleted once the
session has ended.  Hence in ECDHE Private keys will not affect the
confidentiality of the past conversations, albeit part of the initial
handshake. The private key will never aid the attacker as it was only
used to authenticate the server identity If in case the attacker
managed to attain the shared secret, he can only decrypt messages from
that particular session. No previous or future
sessionswould be compromised.\\

\noindent
We have partitioned the standard EC-DHE handshake implementation in
Libressl and separated it into two components. One network facing
component is deemed untrusted and is running as a part of nginx web
server process. The other part resides inside the trusted SGX enclave
and runs as an independent process.  Both the components of the server
communicate via a list of fine grained API’s
wrapped over a named pipe IPC interface.\\

\noindent
The handshake process is carried out in two different phases, one to
authenticate the server identity and other to establish the pre master
secret.  Figure~\ref{fig:ecdhe_handshake}) illustrates the messages
exchanges between client and server’s trusted and untrusted compartments.\\

\noindent
A Https client can initiate the handshake by sending a “client hello”
enclosed with a random number and a list of Elliptic curves it
supports. Upon receiving this request, the nginx server initialises a
new session in the enclave (TODO: Add info related to session
cache/resumption.?) and configures the random number received from the
client. The server random seed is generated inside the enclave. This
will accord to additional security to the system by preventing an
adversary from exploiting the untrusted component of the server and by
influencing a session key generation based on random seeds of his
choice. Enclave code maintains separate state for each session and
will not accept server random as an input argument in any API
interfaces. After obtaining the server random, it is sent to the
client as part of the “Server Hello” message together with Server
Public key certificate and a list of server preferred ciphers.

\begin{figure}[H]
  \centering
  \includegraphics[scale=0.4]{EC-DHE-Handshake.png}
  \caption{EC-DHE Handshake}
  \label{fig:ecdhe_handshake}
\end{figure}

\noindent
In the second phase of DH key exchange, the server will pick one of
the client supported elliptic curves and send the curve id to the
enclave. In the enclave we will generate both the public and private
part of the EC-DH parameters based on the chosen curve id. Only the
public part of the DH param is exposed to the nginx server process and
the private part is kept secret within the enclave. This ensures
additional security to the system by making the enclave completely
abstain from accepting any server side seeds pertaining the shared
secret generation.\\

\noindent
In order to prove, that the server has control of the private key and
DH params, the Server has to produce a digital signature. In our
design this is computed within the enclave. This will include Client
Random, Server Random and Public DH Params signed using server private
key. Server sends the ECDHE parameter (in clear) together with
Signature to the client as part of the “Server Key Exchange”
message. Client verifies this message using the server public key. If
the verification is successful , the client generates its part of EC
params and sends it to the server as part of “Client Key Exchange”
message.\\

\noindent
After receiving the “Client Key Exchange” message, the server will
attain all the required seeds to launch the shared key generation
process. The keys are calculated in a three level process. First a
Pre-Master key is computed from Client Public ECDHE param and Server
ECDHE params(both Private and Public).  The pre-master is then used as
one of the seed along with Client Random and Server Random for the
shared Master secret generation. Server will finally generate the
symmetric session keys with the associated MACs and Initialisation
vectors from the master secret. All the three steps of key
generation processes is securely performed within the enclave.\\

\noindent
The client will then send a “Change Cipher Spec” message to indicate
it will use the newly generated keys to hash and encrypt further
messages. The handshake is complete from the client end after sending
a “Finished” message containing a hash and MAC over all the previous
handshake messages. Upon receiving client finished, the server will
decrypt it using the session keys and validate the integrity of the
handshake process. If the decryption or verification fails, the
handshake is deemed to be failed and all the allocated session
resources are freed. If successful, the server sends a “Change Cipher
Spec” message indicating all messages send from now on will be
encrypted using the shared session secret keys. Finally server
computes its authenticated and encrypted “Finished” Message and sends
it to the Client. This completes the full Handshake process.

\end{document}
