\documentclass[../main.tex]{subfiles}
\begin{document}
\label{sec:ssloverview}
Before discussing previously proposed solutions to the problem we
identified in the Introdcution, we present here a generic overview of
SSL/TLS.

SSL/TLS supports a wide array of ciphers, but they can be roughly
split into two categories: ciphers that support forward secrecy, and
ciphers that do not. Ciphers that offer forward secrecy are ones
wherein a compromised long-term private key does not allow an
adversary to compromise previously eavesdropped, stored
sessions. SSL/TLS's session establishment mechanism is different based
on the category to which the cipher belongs. Consequentially,
depending on the type of cipher used for a given session, our system
must adapt to two, different, control flows. In this section we present
a generalisation of the SSL/TLS handshake that we heavily referred
to in designing our system.

In general, regardless of the cipher used, SSL/TLS utilises
the long-term private key to negotiate ephemeral keys for use in the
current session; hence, to secure the long-term private key, we need
to examine the session establishment mechanism. Roughly, SSL/TLS's
handshake can be split into four steps:

\begin{enumerate}
  \item \textbf{Contacting the server, and establishing parameters for
    the session}. Specifically, the exchange of hello messages that
    include: the server's certificate(s) a list of ciphers supported by
    each side (to determine which cipher is to be used for this session),
    \srandom, and \crandom. The random variables are used to derive the
    secret, used to secure communications.
  \item \textbf{Asymmetric key exchange}: based on the cipher
    selected, the client and server determine the asymmetric keys that are
    to be used to exchange the symmetric keys for the session. This step
    is necessary in two scenarios:
    \begin{enumerate}
      \item The client and server both possess public key
        certificates, and both entities wish to verify each other's identity
        during the handshake. We do not consider this case due to its
        rarity. Generally, the client verifies the server's identity as part
        of the SSL/TLS handshake, and the server verifies the identity of the
        client via some other means, such as a username \& password.
      \item The client and server selected a cipher that offers
        forward secrecy. These ciphers function by first exchanging an
        ephemeral asymmetric secret. This secret is then used in negotiating
        the symmetric secret.
    \end{enumerate} In all other cases, the server's asymmetric keys
    are used to negotiate the ephemeral symmetric secret.
  \item \textbf{Ephemeral symmetric key negotiation}: The client and
    server establish the symmetric secret that is used for the current
    session. The symmetric ephemeral keys are calculated as outlined
    below:

    \begin{enumerate}
      \item The \crandom, \srandom, and a value denoted
        \texttt{PremasterSecret} are combined together, through use of a
        pseudo-random function (PRF), to generate a value called the
        \texttt{MasterSecret}. The \texttt{MasterSecret} is a 48-byte number,
        and is computed using the method outlined here in all SSL/TLS
        ciphers. In contrast, the \texttt{PremasterSecret} is a random value,
        established as part of this step. Arriving to the value of the
        \texttt{PremasterSecret}, however, depends on the cipher used.
      \item The \texttt{MasterSecret} is then used to generate a key
        block. A session key block consists of:
        \begin{itemize}
          \item \texttt{server\_write\_key}: this key is used by the
            server to encrypt outgoing packets, and by the client to decrypt
            incoming packets.
          \item \texttt{client\_write\_key}: this key is used by the
            client to encrypt outgoing packets, and by the server to decrypt
            incoming packets.
          \item \texttt{server\_mac\_secret}: this key is used by the
            server to compute a MAC over outgoing packets, and by the client to
            verify the MAC over incoming packets.
          \item \texttt{client\_mac\_secret}: this key is used by the
            client to compute a MAC over outgoing packets, and by the server to
            verify the MAC over incoming packets.
          \item \texttt{client\_initialisation\_vector(iv)}: this is not
            a key, but a value used to initialise the symmetric cipher at the
            client, before invoking the encryption routine on outgoing packets, and
            at the server, before invoking the decryption routine on incoming packets.
          \item \texttt{server\_iv}: same as above, but used by the
            server before invoking the encryption routine on outgoing packets, and
            by the client before invoking the decryption routine on incoming
            packets.
        \end{itemize}
    \end{enumerate}
  \item \textbf{Verifying the integrity of the just-negotiated keys,
    completing the session establishment}: Both the server and the client
    compute a MAC across the packets exchanged in establishing the
    session. The resultant finished message is encrypted using the
    just-negotiated keys. If both sides successfully verify the MAC, the
    handshake is complete and the session is established. If either side
    fails to verify the MAC, the session is terminated.
\end{enumerate} Figure~\ref{fig:abshandshake} illustrates the
generalization of the SSL/TLS handshake we presented here.

\begin{figure}[H]
  \centering
  \includegraphics[scale=0.4]{images/abstract-handshake.pdf}
  \caption{SSL/TLS handshake generalization}
  \label{fig:abshandshake}
\end{figure}

\end{document}
