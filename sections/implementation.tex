% TODO: Add details about SSL session key management and book keeping in the
% enclave
%NGINX achieves state of the art performance in commodity hardware due to its
%event-driven design which renders the web server capable of scaling to
%hundends of thousands of concurrent connections. This architecture introduced
%an implication in our design regarding handling concurrent SSL handshakes.
%Although an implementation detail, we explain further a simple scheme we used
%for supporting NGINX's behaviour, as it was a crucial part of our final
%performance evaluation results\footnote{Not dealing with concurrent
%connections, would mean that the enclave would be only be able to handle a
%single connection at a time which would end up in severe performance penalties
%in our results.}. According to an event-driven architecture, instead of
%blocking until an I/O operation completes, the web server could switch between
%ongoing connections maximizing this way the system's resource utilization.
%Designing the enclave to store SSL state specific values in a flat memory space
%would result in loosing the state of an ongoing SSL handshake when the web
%server switches to another connection. Our scheme uses a native implementation
%of a dynamic hash table that provides type checking and storing of arbitrary
%data structures. We used this data structure in order to map the state of
%ongoing SSL handshakes or established connections against an identifier and
%pay nearly constant time cost when an enclave tries to retrieve the state of
%an ongoing connection. Because the SSL session ID is considered established
%only when the handshake successfully completes, we assign an SGX specific
%identifier to an in-progress SSL handshake and switch the mapping key within
%the enclave to the SSL session ID when the handshake completes. Note that,
%this mechanism not only enables the enclave to deal with concurrent
%connections, but also lets us support SSL session resumption. In the interest
%of time, we left support of SessionTickets as future work.


\documentclass[../main.tex]{subfiles}

\begin{document}

In this section, we discuss the components of the system we
implemented and the modifications we made to each. We built a
prototype of our design which consists of the following parts: NGINX,
an event-driven web server achieving state-of-the-art performance,
LibreSSL, an implementation of the SSL/TLS protocol, OpenSGX, an
emulation environment of \Intel's SGX hardware, and SGXBridge, an
interface that enables communication between an \textit{enclave
  program} and the untrusted component using. Figure
\ref{fig:implementation-overview} depicts an overview of our prototype
system.

\begin{figure}[H]
  \centering
  \includegraphics[scale=0.25]{images/implementation.png}
  \caption{Secure Web-server block diagram}
  \label{fig:implementation-overview}
\end{figure}

\subsection{NGINX}

NGINX is an open source web server that uses an asynchronous
event-driven approach for handling requests from the clients. The
rationale behind choosing NGINX over Apache is the performance
improvement it offers with its event-driven architecture. In NGINX,
each worker process handles thousands of connections simultaneously as
opposed to Apache's thread per connection approach. Furthermore, NGINX
is lightweight, scalable and also gaining lot of market share compared
to Apache.

The NGINX server process can manage multiple clients simultaneously
and the request events are handled individually as they arrive. The
server is capable to offload the tasks to multiple worker processes in
case of heavy incoming traffic. This architecture introduced two
implications in our design regarding handling concurrent SSL
handshakes and therefore concurrent connections.

First, OpenSGX does not support multi-threading, a fact that
immediately renders the enclave incapable of handling concurrent
requests. We initially considered an alternative design to provide
multi-core support, in which an enclave is launched for each worker
and each enclave communicates with a shared enclave to store/retrieve
state specific to an active SSL handshake/session. Though such a
design would enable our system to support multiple NGINX workers, it
does not present a realistic approach for a production system with
real SGX hardware where multi-threading is natively supported. Thus we
limit our current model to a single core, and in the lack of
possession of real SGX hardware we leave multi-core support as future
work.

Second, to maximise performance, a web server, even on a single core
machine, will try to switch between ongoing connections while it waits
for network I/O. As a result, to achieve good performance, the
\enclaveprogram~must support multiple, concurrent SSL/TLS connections.
Consequently, the \enclaveprogram~ must maintain a mapping between an
identifier and SSL/TLS connections, and the untrusted component can
use the identifier to indicate to the \enclaveprogram~which session an
operation corresponds to. A natural choice for the identifier would be
the SSL/TLS session ID which is established after the SSL/TLS
handshake completes. However, we also required an identifier to
distinguish between SSL/TLS sessions that are still in the handshake
phase; a consequence of the \textit{enclave program} performing
operations \textit{during} session establishment.

To resolve this issue, we assigned a temporary identifier to an
on-going handshake, and used this ID as a mapping key to distinguish
between connections within the \textit{enclave program}. We used a
dynamic hash-table to store state specific to every connection. Upon
successful completion of the SSL/TLS handshake the mapping key of the
hash table is switched to the SSL/TLS session ID. Figure
\ref{fig:session-management} illustrates this process. Note that this
mechanism not only enables the enclave to deal with concurrent SSL
handshakes, but also lets us support SSL/TLS session resumption
through session ID. Due to time constraints, we left support of
Session Tickets, another mechanism used for session resumption, as
future work.

\begin{figure}[H]
  \centering
  \includegraphics[scale=0.25]{images/session-management.png}
  \caption{Session ID establishment process}
  \label{fig:session-management}
\end{figure}

\subsection{LibreSSL}
\label{subsec:libressl}
We used LibreSSL as the SSL/TLS protocol library of choice for our
system. It is a fork of OpenSSL aiming to provide a more secure
implementation and it has undergone major code pruning. Most of the
changes compared to legacy OpenSSL are related to the security
vulnerabilities found in the latter. In our prototype system, both
NGINX and SGX runtime are linked to this library.

% TODO: proof read this paragraph and fix. i.e. not all crypto operations are
% replaced (think of the session keys outside case), some non crypto are
% replcaed too (server random generation). is it correct to say we assume
% untrusted code contains voulnerabilities and is compromised? maybe its just
% not the right place to mention this in the implementation section?
The LibreSSL variant attached to the NGINX process has been modified,
such that the implementation of all crypto operations are replaced
with IPC read()/write() stubs to SGXBridge
(Section~\ref{subsec:sgxbridge}). That is, all the secure operations
of an SSL handshake such as signing, verification, key generation,
encryption or decryption, will invoke an IPC request to the SGX
counterpart. We made careful cuts in LibreSSL's code in order to
minimise the code executed within the trusted component, thus
effectively reducing the size of the TCB. Most of the code executing
within the enclave includes cryptographic operations, except for a
small volume of code responsible for SSL state bookkeeping
reasons\footnote{As mentioned in Section \ref{sec:design}, upon
  exchange of the \texttt{ChangeCipherSpec} messages the enclave
  prepares the environment for handling crypto operations specific to
  the selected cipher.}. The rest of LibreSSL's code that handles
tasks such as connection establishment/termination and state
manipulation, executes outside of the enclave and is assumed to
contain vulnerabilities and be potentially compromised.

\subsection{OpenSGX}
\label{subsec:opensgx}
As described in Section~\ref{sec:opensgx}, OpenSGX is an emulation
environment of \Intel~SGX. We used its runtime to build a prototype of
the designs description in Section~\ref{sec:design}. OpenSGX, however,
does not provide a complete implementation of SGX. Consequentially,
there were a few implementation decisions that were directly affected
by this limitation.

Specifically, OpenSGX does not support multiple threads within an
enclave. This means that, to support multiple server worker processes,
each worker would have to talk to a different enclave, otherwise, a
worker process would be blocked waiting for another worker's request
to complete (something that is not the case if a non-modified server
is in-use). We discuss this issue further in
Section~\ref{sec:futurework}.

Moreover, OpenSGX does not provide a way to programmatically invoke an
enclave from an untrusted component. To circumvent this limitation, we
implemented our design as two processes, one for the untrusted
component, and one for the \textit{enclave program}. The two processes
communicate using a named pipe. We used blocking
\texttt{read()}/\texttt{write()} operations to emulate the untrusted
component switching contexts to execute the \textit{enclave program}.

Finally, OpenSGX does not provide a way of accurately benchmarking
CPU cycle counts because it does not account for the number of cycles
each SGX instruction requires. This led to us estimating cycle counts
for an SGX instruction for the performance evaluation (see
Section~\ref{sec:perfeval}).


% TODO: Mention issues with the provisioning mechanishm TODO: Mention
% returning of IV/nonce to the untrusted part TODO: OpenSGX, anything
% interesting that not mentioned in the previous sections ???

% It is interesting to note that in the SGX part, we maintain a
% session cache based on a dynamic hash table ( SSL - LHASH routines )
% and this distinguishes multiple ongoing sessions with their
% corresponding session ids. This empowers us not only to handle
% concurrent SSL handshakes but also deal with the SSL session
% resumption within the enclave. (More details on cache
% implementation.. John? )

% During the first phase of development, we isolated all the private
% key related operations into the SGX component and returned only
% master key to the untrusted part of the server. To further enhance
% the security, we moved the symmetric session key generation as well
% to the SGX enclave. This brings all the cryptographic operations
% within the secure compartment. It's interesting to note that, after
% the key block generation we had to return IV/nonce and key-block
% length back to the untrusted part of the server. These parameters
% are necessary for changing cipher states in the untrusted LibreSSL
% part. (more to expand by john ??)
%

\subsection{SGXBridge}
\label{subsec:sgxbridge}

SGXBridge is a IPC interface, using named pipes, developed for
exchanging data between the trusted and untrusted process. It declares
all the functions and data structures required by the two components
for data serialisation. Each operation that goes through SGXBridge
requires two \texttt{memcpy()}s and two I/O operations on the named
pipe\footnote{All I/O operations on a named pipe occur on kernel
  buffers whose size may vary from 4K to 64K depending on the kernel
  distribution, and do not incur any actual disk I/O.}. It is worth
mentioning that in the current version of OpenSGX (v1.0), the
interface will read/write data to/from the stub (shared memory) in
buffers of 512 bytes. This immediately negates a pipe buffer larger
than this size, thus we designed SGXBridge to follow this assumption.
We mention though, this does not affect our performance results, as
our evaluation environment executes outside of OpenSGX's runtime. For
simplicity, we designed SGXBridge \texttt{read()}/\texttt{write()}
operations to follow a similar API to UNIX's I/O interface. For
brevity reasons, we omit the implementation details and we refer the
reader to the code.

Using real SGX hardware, the \textit{enclave program} could exist
within a single process's virtual memory (OpenSGX does not provide a
way to programmatically invoke an \textit{enclave program} from an
untrusted process, leading to our implementation with two processes),
and accessed directly without the requirement of an IPC mechanism. We
decided to use these interfaces for our project as it was an IPC
facility already available within the OpenSGX library.

\subsection{Implementation Phases and Tools used}
% \label{subsec:Implementation Phases and Tools used}
Our primary goal during the course of implementing our prototype was
understanding the code flow in LibreSSL, and identifying the specific
functions and data structures that handled the server's long-term
private key. We used a tool called Callgrind to uncover the flow of
execution in NGINX and LibreSSL when an SSL/TLS session is negotiated.
Callgrind dynamically records the sequence of function calls and
provides a snapshot of the runtime behaviour of an application. We
also used a tool called KCachegrind, which presents Callgrind's output
in a graphical manner, making it easier to understand.
 
Using Callgrind's output, we determined that LibreSSL holds a data
structure called \texttt{SSL\_CTX} which is used to create an
\texttt{SSL} object for each connection. The \texttt{SSL_CTX}
structure is the one that holds the long-term private key. The
\texttt{SSL} connection object holds all of the data structures
pertaining to a particular connection, and is used to carry out
SSL/TLS handshakes routines, reads, and writes. In the \textit{enclave
  program} we created a \texttt{SSL\_CTX} data structure, initialising
it with the long-term private key and corresponding public
certificate. We replaced the private key in LibreSSL's
\texttt{SSL_CTX} with a dummy value to prevent crashes. Furthermore,
we maintain an \texttt{SSL} connection object in the \textit{enclave
  program} for each on-going connection.
 
Most of the abstract level functions pertaining to private/session key
isolation were defined in following source files from Libressl
library:
\begin{enumerate}
  \item \texttt{s3\_srvr.c}
  \item \texttt{t1\_enc.c}
\end{enumerate}
 
When NGINX receives a new HTTPS connection, it invokes the
\texttt{SSL\_accept()} which causes the LibreSSL to perform the server
side of the SSL/TLS handshake.\texttt{SSL_accept()} blocks until the
handshake is complete. We performed a comprehensive review of every
switch-case within this API, and identified various internal functions
(and associated data structures) invoked during the different steps
involved in a handshake. At appropriate points within the
sub-functions we replaced operations with stubs that invoked the
SGXBridge interface, passing the inputs we specified in the design
section. For instance, after receiving a \texttt{client\_hello}
message, LibreSSL loads \crandom~to an \texttt{SSL} connection object
and generates \srandom. We have replaced this code such that it sends
\crandom~ to the \enclaveprogram, by invoking the interface that
generates \srandom, and reads the \srandom~that is generated by the
\enclaveprogram. We identified other places within LibreSSL's code
that require the long-term private key, and replaced accesses to
long-term private key with stubs that send data to the
\enclaveprogram~to perform operations such as decrypting, signing,
etc.

At a later stage, during the course of the project, we created the
design where session keys are held by the \enclaveprogram. To
implement this, we identified the routines that initialised the data
structures pertaining to a particular cipher suite after completing
establishing an SSL/TLS session, and the specific functions used to
perform encryption and decryption operations. In the enclave, we
reproduced this functionality, removing it from LibreSSL. In our
implementation for the design where session keys are outside the
enclave, we support the legacy RSA cipher suites, such AES256-SHA, and
cipher suites with forward secrecy support such as ECDHE-RSA-AES256
GCM-SHA384. In our design where session keys are inside the enclave,
we support AEAD (Authenticated Encryption with Associated Data) cipher
suites. All AEAD cipher suites use AES in GCM (Galois Counter Mode)
for symmetric cryptography. GCM is a high performance mode of
operation for AES, yielding state-of-the-art performance in commodity
hardware by using an instruction or hardware pipeline, as opposed to
CBC mode which incurs significant pipeline stalls.

It is interesting to note that at various points we used GDB to break
the execution and step through the code flow. We developed and
maintained a single bash script that automates the building and
installation of all the software packages used within the system. For
convenience, we add two compilation flags in LibreSSL:
\texttt{--enable-sgx} and \texttt{--enable-sgx-keyblock}. The former
enables the scenario where only the server's private key is protected,
while the latter enables the scenario where both the server's private
key and the SSL session key-block are secured by the enclave.

\end{document}

%%% Local Variables:
%%% mode: latex
%%% TeX-master: "../main"
%%% End:
