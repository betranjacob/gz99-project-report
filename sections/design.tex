\documentclass[../main.tex]{subfiles}

\begin{document}
	By utilizing the aforementioned remote-attestation process, we can provision the trusted component 
	of the remote server application with an SSL private key while making it extremely difficult* for even
	\textit{privileged} processes running on the cloud provider to access the key. This is a stark contrast to current
	schemes wherein the private key is merely shipped as part of the server application to the cloud provider. 

	As previously highlighted in Section~\ref{sec:ssloverview}, the private key is required by a subset of the operations executed during the handshake step.
	These operations have to be executed within the trusted component, and are invoked by the non-trusted component via an interface that we define in this section. Note that
	the interface has to be carefully designed, allowing the handshake to complete correctly while not exposing the long term private key through an oracle.

	%TODO: introduce this better to make it clear that there are 2 different handshakes; I only went into some detail here, definitely need more depth
	
	Take for example an interface where the non-trusted component supplies \texttt{(ServerRandom, ClientRandom, \{PremasterSecret\}$_K$)}. Such an interface, while maintaining the secrecy of
	the private key's bits, would allow an adversary capable of exploiting the non-trusted component to generate the symmetric keys for previously eavesdropped sessions.
	Therefore, it is no better than leaving the private key in the non-trusted component. However, observe that it is not necessary for \texttt{ServerRandom} to be provided by the non-trusted component,
	it need only be provided by the \textit{server}. 

	%TODO: Cite wedge here somewhere, I do not like how I did it
	We can adjust the interface so that the non-trusted component supplies \texttt{(ClientRandom, \{PremasterSecret\}$_K$)}, both of 
	which are generated by the client, and the trusted component generates a new \texttt{ServerRandom} every time the interface is invoked. The resulting interface
	ensures that, even if a previously eavesdropped \texttt{\{PremasterSecret\}$_K$} is provided, a fresh session-key is computed on every invocation. 
	%Freshness of the generated session-key has been used in [WEDGE CITATION] to maintain the secrecy of an SSL/TLS private key in the face of a passive eavesdropper how can exploit the server application's network facing
	%process. We use it to secure the private key against an adversary who can exploit the underlying operating system and/or a malicious cloud provider.

	\noindent
	\\We consider two scenarios:
	\begin{itemize}
		\item Session keys available to the OS
		\item Session keys hidden inside the enclave and accessible through encrypt/decrypt oracle
	\end{itemize}


	\begin{figure}[H]
		\centering
		\makebox[0pt]{
			\begin{Lsc}{none}{7}{10}
				\begin{lscinst}[noboxSD]{Client}
					\lscLine{hot}{0.5}
					\lscAsynSnd{clnthello}
					\lscLine{hot}{3}
					\lscAsynRcv{srvrhello}
					\lscLine{hot}{0.4}
					\lscAsynRcv{srvrcert}
					\lscAsynSnd{clntkeyx}
					\lscLine{hot}{0.4}
					\lscAsynSnd{clntchngcipher}
					\lscLine{hot}{0.4}
					\lscAsynSnd{clntfinish}
					\lscLine{hot}{2.5}
					\lscAsynRcv{srvrchngcipher}
					\lscLine{hot}{0.4}
					\lscAsynRcv{srvrfinish}
					\lscLine{hot}{2.4}
				\end{lscinst}
				\begin{lscinst}[noboxSD]{}
				\end{lscinst}
				\begin{lscinst}[noboxSD]{}
				\end{lscinst}
				\begin{lscinst}[noboxSD]{Server (non-trusted)}
					\lscLine{hot}{1}
					\lscAsynRcv{clnthello}
					\lscAsynSnd{clntrandom}
					\lscLine{hot}{0.4}
					\lscAsynSnd{getsrvrandom}
					\lscLine{hot}{1.6}
					\lscAsynRcv{srvrandom}
					\lscAsynSnd{srvrhello}
					\lscLine{hot}{0.4}
					\lscAsynSnd{srvrcert}
					\lscLine{hot}{1}
					\lscAsynRcv{clntkeyx}
					\lscLine{hot}{0.4}
					\lscAsynRcv{clntchngcipher}
					\lscLine{hot}{0.4}
					\lscAsynRcv{clntfinish}
					\lscAsynSnd{getsesskey}
					\lscLine{hot}{1.5}
					\lscAsynRcv{sesskey}
					\lscAsynSnd{srvrchngcipher}
					\lscLine{hot}{0.4}
					\lscAsynSnd{srvrfinish}
					\lscLine{hot}{2.9}
				\end{lscinst}
				\begin{lscinst}[noboxSD]{}
				\end{lscinst}
				\begin{lscinst}[noboxSD]{}
				\end{lscinst}
				\begin{lscinst}[noboxSD]{Server (trusted)}
					\lscLine{hot}{1.5}
					\lscAsynRcv{clntrandom}
					\lscLine{hot}{0.4}
					\lscAsynRcv{getsrvrandom}
					\lscLine{hot}{0.6}
					\lscAsynSnd{srvrandom}
					\lscLine{hot}{3.2}
					\lscAsynRcv{getsesskey}
					\lscLine{hot}{0.5}
					\lscAsynSnd{sesskey}
					\lscLine{hot}{3.8}
				\end{lscinst}
				\lscAsyn{clnthello}{hot}{\texttt{ClientHello\{client\_random\}}}
				\lscAsyn{clntrandom}{hot}{\texttt{ClientRandom}}
				\lscAsyn{getsrvrandom}{hot}{\texttt{GetServerRandom}}
				\lscAsyn{srvrandom}{hot}{\texttt{ServerRandom}}
				\lscAsyn{srvrhello}{hot}{\texttt{ServerHello\{server\_random\}}}
				\lscAsyn{clntkeyx}{hot}{\texttt{ClientKeyExchange\{premaster\_secret\}$_K$}}
				\lscAsyn{clntchngcipher}{hot}{\texttt{ClientChangeCipherSpec}}
				\lscAsyn{clntfinish}{hot}{\texttt{MAC<master\_secret, all messages>}}
				\lscAsyn{getsesskey}{hot}{\texttt{GetSessionKeys\{premaster\_secret\}$_K$}}
				\lscAsyn{sesskey}{hot}{\texttt{SessionKeyBlock}}
				\lscAsyn{srvrfinish}{hot}{\texttt{MAC<master\_secret, all messages>}}
				\lscAsyn{srvrchngcipher}{hot}{\texttt{ServerChangeCipherSpec}}
			\end{Lsc}
		}
		\caption{SSL handshake}
		\label{fig:rsashake}
	\end{figure}



	\paragraph{ECDHE Handshake}
	TBC..

\end{document}