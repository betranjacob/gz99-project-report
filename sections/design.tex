\documentclass[../main.tex]{subfiles}
 
\begin{document}
\setlength\parindent{0pt}

\subsection{Securing SSL handshake}
Once the RSA private key is provisioned to a trusted application the sensitive
operations performed during the handshake can be moved inside the enclave.
This section discusses these operations and the interface between non- and
trusted codebase for both RSA and ECDHE cipher suite families.
The ultimate goal is to be able to generate the symmetric session keys without
exposing the long term private key or providing encrypt/decrypt oracle.

\paragraph{RSA Handshake}
In order to generate the master key from which further (session) keys
are derived RSA handshake uses three random components:
\begin{itemize}
	\item Client random - generated by the client and sent over the network in
	clear
	\item Premaster secret - generated by the client and sent over the network
	encrypted with the servers public key
	\item Server random - generated by the server and sent in clear
\end{itemize}


The secrecy of the master key is guaranteed by the fact that only the server is in possesion of the private key to decrypt the premaster secret.

We need to secure against the malicious OS supplying the server random component, which would force the trusted component to generate the same session keys which would lead to private key oracle. [Wedge citation]


We consider two scenarios:
\begin{itemize}
	\item Session keys available to the OS
	\item Session keys hidden inside the enclave and accessible through encrypt/decrypt oracle
\end{itemize}


\begin{figure}[H]
	\centering
	\begin{Lsc}{none}{5}{10}
		\begin{lscinst}[noboxSD]{Client}
			\lscLine{hot}{0.5}
			\lscAsynSnd{clnthello}
			\lscLine{hot}{3}
			\lscAsynRcv{srvrhello}
			\lscLine{hot}{0.4}
			\lscAsynRcv{srvrcert}
			\lscAsynSnd{clntkeyx}
			\lscLine{hot}{0.4}
			\lscAsynSnd{clntchngcipher}
			\lscLine{hot}{0.4}
			\lscAsynSnd{clntfinish}
			\lscLine{hot}{2.5}
			\lscAsynRcv{srvrchngcipher}
			\lscLine{hot}{0.4}
			\lscAsynRcv{srvrfinish}
			\lscLine{hot}{2.4}
		\end{lscinst}
		\begin{lscinst}[noboxSD]{}
		\end{lscinst}
		\begin{lscinst}[noboxSD]{Server}
			\lscLine{hot}{1}
			\lscAsynRcv{clnthello}
			\lscAsynSnd{clntrandom}
			\lscLine{hot}{0.4}
			\lscAsynSnd{getsrvrandom}
			\lscLine{hot}{1.6}
			\lscAsynRcv{srvrandom}
			\lscAsynSnd{srvrhello}
			\lscLine{hot}{0.4}
			\lscAsynSnd{srvrcert}
			\lscLine{hot}{1}
			\lscAsynRcv{clntkeyx}
			\lscLine{hot}{0.4}
			\lscAsynRcv{clntchngcipher}
			\lscLine{hot}{0.4}
			\lscAsynRcv{clntfinish}
			\lscAsynSnd{getsesskey}
			\lscLine{hot}{1.5}
			\lscAsynRcv{sesskey}
			\lscAsynSnd{srvrchngcipher}
			\lscLine{hot}{0.4}
			\lscAsynSnd{srvrfinish}
			\lscLine{hot}{2.9}
		\end{lscinst}
		\begin{lscinst}[noboxSD]{}
		\end{lscinst}
		\begin{lscinst}[noboxSD]{Enclave}
			\lscLine{hot}{1.5}
			\lscAsynRcv{clntrandom}
			\lscLine{hot}{0.4}
			\lscAsynRcv{getsrvrandom}
			\lscLine{hot}{0.6}
			\lscAsynSnd{srvrandom}
			\lscLine{hot}{3.2}
			\lscAsynRcv{getsesskey}
			\lscLine{hot}{0.5}
			\lscAsynSnd{sesskey}
			\lscLine{hot}{3.8}
		\end{lscinst}
		\lscAsyn{clnthello}{hot}{\texttt{Client Hello\{client\_random\}}}
		\lscAsyn{clntrandom}{hot}{\texttt{ClientRandom}}
		\lscAsyn{getsrvrandom}{hot}{\texttt{GetServerRandom}}
		\lscAsyn{srvrandom}{hot}{\texttt{ServerRandom}}
		\lscAsyn{srvrhello}{hot}{\texttt{ServerHello\{server\_random\}}}
		\lscAsyn{clntkeyx}{hot}{\texttt{ClientKeyExchange\{premaster\_secret\}}}
		\lscAsyn{clntchngcipher}{hot}{\texttt{ClientChangeCipherSpec}}
		\lscAsyn{clntfinish}{hot}{\texttt{MAC<master\_secret, all messages>}}
		\lscAsyn{getsesskey}{hot}{\texttt{GetSessionKeys\{premaster\_secret\}}}
		\lscAsyn{sesskey}{hot}{\texttt{SessionKeyBlock}}
		\lscAsyn{srvrfinish}{hot}{\texttt{MAC<master\_secret, all messages>}}
		\lscAsyn{srvrchngcipher}{hot}{\texttt{ServerChangeCipherSpec}}
	\end{Lsc}
\end{figure}



\paragraph{ECDHE Handshake}
TBC..

\end{document}