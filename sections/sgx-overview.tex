\documentclass[../main.tex]{subfiles}

\begin{document}
% TODO: Document SGX instructions EENTER, ERESUME and EEXIT
\subsection{An overview of \Intel SGX}

This section will not cover all of the details of SGX, but only those
applicable to our project; for a complete treatment of SGX please
refer to~\cite{IntelCorporation2010}. \Intel SGX is a set of x86
instructions that allow for a programming model wherein a program can
be split into two components: an untrusted component that executes as
normal and a trusted component that executes within a protected area
of RAM, called an enclave, which can only be accessed when executing
the trusted component.

The protection of an enclave is managed by the CPU; any data written
to the enclave is encrypted first by a memory encryption engine
(implemented in hardware) and is only decrypted when required by the
CPU during the execution of the trusted component for which that
enclave belongs. The key used for this encryption process is derived
from a combination of a device key, unique to each SGX-enabled CPU and
the ``identity'' of the enclave called MRENCLAVE, a cryptographic hash
of the enclave's contents at the trusted component's
initialization. SGX thus ensures that no process other than the one
that initialized the enclave can access the protected area.

Interacting with the trusted component, as a result, may only occur
through invoking a programmer defined interface, called a callgate, as
depicted in Figure~\ref{fig:sgxhighlevel}.

\begin{figure}[H]
  \centering
  \includegraphics[scale=0.25]{images/sgxhighlevel.png}
  \caption{Interaction of the untrusted part of the application with
    the trusted part can only occur through a callgate}
  \label{fig:sgxhighlevel}
\end{figure}
%%%%%%%%%%%%% transition to OpenSGX still have to add more to the SGX
%%%%%%%%%%%%% overview

At the time we begun working on this project we did not posses access
to SGX hardware, forcing us to use a simulator. There were two
choices, at the time, OpenSGX and the \Intel~ Windows SDK's simulation
mode. We selected the former to implement our prototype as it was
compatible with Linux, a platform we were more familiar with, and had
several examples we could refer to for guidance in our implementation
(the Intel SDK, at the time, was under-documented). The following
section provides a quick examination of OpenSGX's salient features.
\end{document}
