\documentclass[../main.tex]{subfiles}
 
\begin{document}

We developed our prototype using OpenSGX which, as detailed in
Section~\ref{sec:background}, is an emulator for the SGX instruction
set and relies on QEMU to execute the trusted code. Consequentially,
we cannot measure end to end performance by simply executing our
prototype and tabulating how many requests per second are completed,
due to the overhead of emulation. Instead, we resorted to modeling the
performance of the prototype in a manner similar to~\cite{Baumann14}.

The remainder of this section describes the model we implemented to
carry out our performance measurements and details the results we
acquired from running our tests.

\subsection{Performance Model}

To model the performance of our prototype we implemented a second
version of our enclave program that has no dependencies on OpenSGX and
replaces certain SGX instructions with busy waits, simulating the
overhead in executing them. In dong so, we make the same assumptions
as~\cite{Baumann14} namely:
\begin{enumerate}
  \item We assume that the CPU used in testing performs the same as an
    SGX-enabled CPU for all non-SGX instructions
  \item We assume that the EPC is large enough to accommodate the
    entirety of the trusted component and any data structures created
    after program start-up
\end{enumerate}
Leaving the overhead of SGX instructions, memory encryption, and
asynchronous exits unaccounted for.

SGX instructions that bootstrap the enclave and verify its integrity
are only executed once at startup, and therefore have no effect on
runtime performance. As a result, we only simulate the overhead of
EENTER, ERESUME and EEXIT. Simulating the slow down in memory accesses
due to the need for decrypting RAM contents before being able to
access them in processors cache, was carried out by reducing the
memory's clock in~\cite{Baumann14}. Such a proxy works for their
system because it executes within an enclave in its entirety. Clearly,
such an approach would underestimate the performance of our system
because only the trusted component incurs the overhead of memory
encryption. Instead we captured the number of Last Layer Cache (LLC)
misses as a relevant measure.

Finally, the performance of `malloc` is slower for our test program
than what it would be within real SGX because we have to context
switch to the OS to allocate memory.

\subsection{Hypothesis}
We expect to see the most performance slowdown due to expensive
context switches between untrusted and enclave programs. This is due
to the required TLB flush (??) accompanying EENTER and EEXIT
instructions.

For the case when master secret and session keys are available to the
OS we don't expect significant slowdown as it only adds 4 context
switches during the handshake that can be amortised over multiple
connections.

In contrast, the second case, providing stronger security guarantees,
should perform considerably poorer due to extra XXX context switches
during the handshake. Moreover this scenario incurs additional
slowdown due to the extra 4 context switches for every request made to
the server.

We expect size of the static content served not to influence
performance of the session keys outside version and degrade linearily
(???) in case of session keys inside the enclave.

The performance of ECDHE handshake should be worse than that of RSA
handshake due to the need for deriving fresh asymetric key for each
new session, however, the drop in performance of our system compared
to vanilla nginx should be roughly constant regardless of the type of
the key exchange mechanism.

\subsection{Microbenchmarks}
In order to gain better insight about our implementation we measured
its following aspects.

OpenSGX can report the performance measurement of an enclave program
which includes the number of kernel switches and number and type of
enclave instructions executed. We report the relevant number of
instructions per request.

To be able to reason about performance hit due to memory decryption we
captured the number of LLC misses using Intel VTune Amplifier. We
tried to verify its reported values using perf tool, but chose the
former because it had all CPU specific counters predefined in the
program and was available under student license. perf also lacked
thourough documentation and would require consulting Intel Reference
Manual for the counter values for our CPU.
 
To measure the size of the TCB of the enclave program we generated its
static call graph (using Egypt) to determine all the libressl entry
point functions and how they follow internally. We then manually
copied the source of all functions used and counted the resulting SLOC
(with cloc). We also report the number of changes we have made to the
libressl and nginx source code.

\subsection{Tests setup}
We run our tests on a Amazon EC2 m4.large instances, with the server
and client located on different virtual machines within the same data
center. These machines have 2 virtual CPUs with 8 GiB RAM and
``moderate'' network performance, which according to [REFERENCE -
http://stackoverflow.com/a/35806587 or
http://stackoverflow.com/questions/5257553/coloring-white-space-in-git-diffs-output]
corresponds to 450 (2011) or 790 (2015) Mbit/s. Our own tests using
iperf between the client and server machines reported 465 Mbit/s. The
underlying hardware is Intel Xeon E5-2676 v3 clocked @2.4 GHz
\cite{aws_instances}. We use Ubuntu 14.04 with 4.4.0-34-generic kernel
for our OS and a custom script using \textit{httperf} to generate the
load and measure the performance.

We have measured the end to end performance for ciphersuites not
offering forward secrecy based on RSA handshake as well as suites
providing forward secrecy based on ECDHE:
\begin{enumerate}
  \item (RSA-)AES256-GCM-SHA384
  \item ECDHE-RSA-AES256-GCM-SHA384
\end{enumerate}

We use the AES as it can benefit from hardware acceleration using
AESNI instructions present in modern devices.

We tested two versions of our program corresponding to the two threat
models discussed in Section~\ref{sec:design}:
\begin{enumerate}
  \item session keys available to the OS
  \item session keys hidden from the OS
\end{enumerate}

We have tested each of the above setups with a range of static
document sizes (1 KB - 10 MB) and a range of values for the busywait
instruction delays (10, 30, 50 K) - a subset of values used
by~\cite{Baumann14})

We performed the tests on the busywait instrumented enclave program
communicating with with sgx instrumented libressl 2.4.1 statically
linked to nginx 1.11.1. We ran the same set of tests with an
unmodified version of nginx+libressl for baseline comparison.

We also run the tests inside OpenSGX to learn which and how many SGX
instructions were executed and to be able to verify our performance
results.

As we were unable to obtain data from real life scenarios, we also
varied the ratio betewen new and reused (cached) sessions between 0-30
\% as well as measuring the performance without session caching (100
\% new connections) to capture the worst case scenario.

\subsection{Results}

Table XXX presents the number of modifications to the source code. The
enclave program TCB is XXX of sloc.

We used the formula from [page 30 in
\url{https://software.intel.com/sites/landingpage/legacy/pdfs/Using_Intel_VTune_Amplifier_XE_on_2nd_Gen_Intel_Core_Family.pdf}]
to convert the number of LLC misses to the percentage of cycles the
enclave program spent on waiting for cache misses. We multiply this by
the memory overhead reported in [CITATION NEEDED] (5-14 \%)

Additionally we report that this not a performance bottleneck as it is
under 0.2\% which Intel considers `normal case`.

\begin{table}[H]
  \resizebox{\linewidth}{!}{ \pgfplotstabletypeset[assign column
    name/.style={/pgfplots/table/column name={\textbf{#1}}},col
    sep=comma, header=has colnames,columns/test/.style={postproc cell
      content/.append style={ /pgfplots/table/@cell
        content/.add={$\bf}{$},
      }}]{results/sgx_detail_instruction_count.csv} }
\end{table}

\begin{figure}[H]
  \centering
  \begin{tikzpicture}
    \begin{axis}[pretty,ylabel = Responses/Second, ymin = 0,
      xlabel = Number Of Busywait Cycles (Cycles * $10^4$), legend pos=outer
      north east,  xtick scale label code/.code = {}]
      \addplot table[y=Key Outside RSA, mark = *] from \cyclereqs;
      \addlegendentry{Session Keys Outside \textit{RSA}}; \addplot
      table[y=Key Outside ECDHE, mark = *] from \cyclereqs;
      \addlegendentry{Session Keys Outside \textit{ECDHE}}; \addplot
      table[y=Key Inside RSA, mark = *] from \cyclereqs;
      \addlegendentry{Session Keys Inside\textit{RSA}}; \addplot
      table[y=Key Inside ECDHE, mark = *] from \cyclereqs;
      \addlegendentry{Session Keys Inside \textit{ECDHE}};
    \end{axis}
  \end{tikzpicture}
\end{figure}
% From wedge: - maximum throughput that the SGX-partitioned version of
% nginx and the original version of nginx can sustain (req/s) - all
% sessions cached vs non cached -

\subsection{Discussion}

\end{document}

%%% Local Variables:
%%% mode: latex
%%% TeX-master: "../main"
%%% End:
