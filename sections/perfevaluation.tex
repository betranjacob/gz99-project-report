\documentclass[../main.tex]{subfiles}
 
\begin{document}

We developed our prototype using OpenSGX which, as detailed in 
Section~\ref{sec:background}, is an emulator for the SGX instruction set and
relies on QEMU to execute the trusted code. Consequentially, we cannot measure
end to end performance by simply executing our prototype and tabulating how
many requests per second are completed, due to the overhead of emulation.
Instead, we resorted to modeling the performance of the prototype in a manner
similar to~\cite{Baumann14}.

% Need to detail hardware used for testing here....
The remainder of this section describes the model we implemented to carry out
our performance measurements and details the results we acquired from running
our tests.

\subsection{Performance Model}

To model the performance of our prototype we implemented a second version of
our enclave program that has no dependencies on OpenSGX and replaces certain
SGX instructions with busy waits, simulating the overhead in executing them.
In dong so, we make the same assumptions as~\cite{Baumann14} namely:
\begin{enumerate}
  \item We assume that the CPU used in testing performs the same as an
    SGX-enabled CPU for all non-SGX instructions
  \item We assume that the EPC is large enough to accommodate the
    entirety of the trusted component and any data structures created
    after program start-up
\end{enumerate}
Leaving the overhead of SGX instructions, memory encryption, and asynchronous
exits unaccounted for.

SGX instructions that bootstrap the enclave and verify its integrity are only
executed once at startup, and therefore have no effect on runtime performance.
As a result, we only simulate the overhead of EENTER, ERESUME and EEXIT.
Simulating the slow down in memory accesses due to the need to decrypt the RAM
contents before being able to access them in processors cache, was carried out
by reducing the memory's clock in~\cite{Baumann14}. Such a proxy works for
their system because it executes within an enclave in its entirety. Clearly,
such an approach would underestimate the performance of our system because
only the trusted component incurs the overhead of memory encryption. We have
not been able to find a solution to model this performance hit in our system.

\subsection{Hypothesis}
We expect to see the most performance slowdown due to expensive context
switches between untrusted and enclave programs. This is due to the required
TLB flush (??) accompanying EENTER and EEXIT instructions.

For the case when master secret and session keys are available to the OS we
don't expect significant slowdown as it only adds 4 context switches during
the handshake that can be amortised over multiple connections.

In contrast, the second case, providing stronger security guarantees, should
perform considerably poorer due to extra XXX context switches during the
handshake. Moreover this scenario incurs additional slowdown due to the extra
4 context switches for every request made to the server.

\subsection{Tests setup}
% hardware specs
We run our tests on a XXX core, XXX GHz machine, with XXX RAM, XXX network
card, running Debian / Ubuntu X.X. To generate the load we used Apache
Benchmark XXX.

We have tested end to end performance for ciphersuites not offering forward
secrecy based on RSA handshake as well as suites providing forward secrecy
based on ECDHE:
\begin{enumerate}
  \item (RSA-)AES256-GCM-SHA384
  \item ECDHE-RSA-AES256-GCM-SHA384
\end{enumerate}

We use the AES as it can benefit from hardware acceleration using AESNI
instructions present in modern devices.

We tested two versions of our program, for the two threat models discussed in
Section~\ref{sec:design}:
\begin{enumerate}
  \item session keys available to the OS
  \item session keys hidden from the OS
\end{enumerate}

For the session keys inside the enclave we have tested a range of static
document sizes to capture the cost of encrypt / decrypt oracle interface.

We performed the test for the busywait instrumented enclave program with sgx
instrumented libressl 2.4.1 statically linked to nginx 1.11.1. We ran the same
set of tests with an unmodified version of nginx+libressl for baseline
comparison.

We also run the test inside OpenSGX to learn which and how many SGX
instructions were executed

\subsection{Results}

% From wedge:
%  - maximum throughput that the SGX-partitioned version of nginx and the
%  original version of nginx can sustain (req/s)
%  - all sessions cached vs non cached
%  - 

\subsection{Discussion}

\end{document}
