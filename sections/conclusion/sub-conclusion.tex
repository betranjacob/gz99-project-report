\documentclass[../main.tex]{subfiles}
 
\begin{document}

Legacy web servers follow a monolithic design, where SSL/TLS's
long-term private key is accessible to all of the application's
components. As a result, if an assailant can compromise any of the web
server's unprivileged components, the operating system, the
hypervisor, or is a malicious cloud provider, they can obtain the
long-term private key. If the long-term private key is compromised,
SSL/TLS's guarantees of confidentiality and integrity are negated. We
proposed a design for web servers that are deployed on the cloud
environment, under the assumption that the cloud provider is malicious
and/or the underlying operating system and hypervisor are untrusted.
Our design uses \Intel~SGX to provide isolation guarantees for the
server's private key for both cipher suites that lack or offer forward
secrecy support. We partitioned a legacy, monolithic web server into a
trusted and an untrusted component. The trusted component was secured
in an SGX container, minimising the TCB to include the trusted
component and the CPU, instead of the entire application, OS,
hypervisor and cloud provider. We augmented the baseline design to
further secure the SSL/TLS session keys, hardening the server against
an adversary that wishes to decrypt recorded sessions at their
leisure. We developed a prototype implementation using NGINX and
LibreSSL and evaluated the overhead \Intel~SGX introduces to the web
server's end-to-end performance. The results verify our initial
hypothesis and show that the overhead is large, which may discourage
people from using our system in production environments under heavy
workloads.\\

\noindent
The source code of our prototype implementation is publicly available at:\\
\url{https://github.com/ahawad/ssl-partition}

\end{document}
