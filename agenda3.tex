\documentclass[11pt,agenda]{meetingmins}
\setcommittee{SGX Project Meeting \#3}
\setdate{July 14, 2016}

\begin{document}
	\maketitle
	\section{What we have done}
		\begin{items}
			\item OpenSGX
			\begin{items}
				\item Remote Attestation sample provided by OpenSGX team is implemented incorrectly. We fixed it and implemented provisioning on top of the fixed version.
			\end{items}
			\item NGINX + LibreSSL
			\begin{items}
			\item Through use of valgrind we identified the following:
				\begin{items}
					\item Function callgraph during handshake 
					\item Private key handling functions
				\end{items}
			\item Implemented and tested a custom NGINX module 
			\item Identified the data structures that contain the private key and the ones used during certificate verification
			\item Attempted to use crowbar, but we could not find the version of \texttt{pin} against which it is compiled and could not run the tool as a result.
			\end{items}
			\item Performance Evaluation
			\begin{items}
				\item Tested Apache JMeter to measure end-to-end performance; however, we concluded that this tool may offer more than is required for our purposes.
				\item Found a second tool called ApacheBench. This is a command-line utility included with Apache that can be used to specify things such as number of users
				and the url to test.
			\end{items}
		\end{items}
	\section{What we want to discuss}
		\begin{items}
			\item NGINX + LibreSSL
			\begin{items}
				\item Regarding ciphers that implement forward secrecy, would they require us to change our partitioning? If so, could we disregard them for now?
			\end{items}
			\item Performance Evaluation
			\begin{items}
				\item OpenSGX implements non-enclave to enclave communication through a pipe-like interface. The overhead of this is far greater than the model used by Intel (RPC-like). 
				\item Another possible metric for end-to-end performance is response time vs. number of users. It maybe an easier metric to measure than requests per second. 
			\end{items}
		\end{items}
	\section{what we are planning to do next week}
		\begin{items}
			\item 
				Some OpenSSL calls will have to happen within the enclave. This would entail either transferring the SSL context (used as an argument for all the function calls)  into the enclave, or
				using an alternative set as functions provided in the enclave through a library called polarSSL.
			\item 
				Start working on integrating SGX+NGINX+LibreSSL
		\end{items}

\end{document}
